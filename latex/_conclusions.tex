Throughout this thesis, we have conducted a thorough investigation and comparison of two fundamental approaches: MixMatch
and the novel symmetrical equilibrium learning algorithm within the context of segmentation tasks.

We acknowledge that our deviation from the original assignment, specifically the absence of experiments conducted on the actual 
time series of multi-spectral satellite data for the national park Bohemian Switzerland, may raise questions regarding the validity 
and applicability of our findings to this specific task. However, we would like to provide a comprehensive explanation for this 
decision and offer reasoning as to why our experiments and obtained results remain valuable and relevant.

Instead of utilizing satellite imagery, we chose to work with the (almost) publicly available CityScape dataset, which is widely
used and established. This decision was motivated by two factors: ensuring reproducibility and saving time. Although the CityScape 
dataset primarily consists of urban street scenes, we believe that the complexity of segmentation in this dataset is comparable to, or
even higher than, the challenges encountered in land-coverage segmentation of forests. Additionally, we consider the convolution 
process, whether in 2D or 3D, to be fundamentally similar from the perspective of the network and training process. Therefore, we 
can extrapolate the results obtained on the CityScape dataset to the task of land cover classification.

In light of these considerations, we focused on exploring and evaluating the effectiveness of MixMatch and Symmetric
Equilibrium Learning for VAE for the segmentation task in general. We aimed to gain insights into their applicability 
and performance by examining their advantages and shortcomings.

Our experiments with MixMatch have demonstrated its efficacy, especially in scenarios where labeled data are scarce. The integration of 
consistency regularization and proxy-labeling methods in MixMatch have proven effective in leveraging unlabeled data to enhance segmentation
performance and promote visually coherent model predictions.

The experiments for symmetrical learning of HVAE have shown that a U-net-like architecture, with slight adaptations, can be employed 
for the segmentation task. Specifically, including global skip connections is crucial in enabling the HVAE decoder to generate consistent images. 
The current results do not outperform the best obtainable supervised baseline, and although there is room for further improvement in the HVAE 
architecture, the current results nevertheless serve as proof of concept. 

The experiments have also raised questions that warrant further investigation to improve the HVAE architecture.
One such question pertains to the observed disregard of color information in the decoded images. Understanding the underlying reasons for 
this phenomenon and its relationship to the performance of the encoder and decoder is essential to improve the overall performance of the HVAE model.