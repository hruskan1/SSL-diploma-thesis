Throughout this thesis, we have conducted a thorough investigation and comparison of two key approaches: MixMatch
and the novel symmetrical equilibrium learning algorithm, within the context of segmentation tasks. In this final chapter, we
will reflect on our findings, discuss the implications of our work, and outline potential avenues for future research.

We acknowledge that our deviation from the original assignment, specifically the absence of experiments conducted on the actual 
time series of multi-spectral satellite data for the national park Bohemian Switzerland, may raise questions regarding the validity 
and applicability of our findings to this specific task. However, we would like to provide a comprehensive explanation for this 
decision and offer reasoning as to why our experiments and resulting findings remain valuable and relevant.

Instead of utilizing satellite imagery, we chose to work with the (almost) publicly available Cityscape dataset, which is widely
used and established. This decision was motivated by two factors: ensuring reproducibility and saving time. Although the Cityscape 
dataset primarily consists of urban street scenes, we believe that the complexity of segmentation in this dataset is comparable to, or
even higher than, the challenges encountered in land-coverage segmentation of forests. Additionally, we consider the convolution 
process, whether in 2D or 3D, to be fundamentally similar from the perspective of the network and training process. Therefore, we 
can extrapolate the results obtained on the Cityscape dataset to the task of land cover classification.

In light of these considerations, we focused on exploring and evaluating the effectiveness of MixMatch and 
Symmetric Equilibrium Learning for VAE for the segmentation task in general. By examining their
advantages and shortcomings, we aimed to gain insights into their applicability and performance.

Our experiments with MixMatch have demonstrated its efficacy, especially in scenarios where labeled data is scarce. The integration of 
consistency regularization and proxy-labeling methods in MixMatch has proven effective in leveraging unlabeled data to enhance segmentation
performance and promote visually coherent model predictions.