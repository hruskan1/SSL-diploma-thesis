\section{Symmetric leaning in VAE}
\label{sec:symmetric_learning}
The authors of~\cite{sym-learning-2023} present an alternative approach to maximizing the evidence lower bound (ELBO)
in Variational Autoencoders (VAEs). The traditional ELBO optimization imposes restrictions on the architectures of 
VAEs, as it requires the latent distributions to be in closed form while only providing data samples. This asymmetry in
the ELBO formulation contributes to the issue of blurriness in generated images (discussed in~\ref{item:blurriness-of-img}),
which has been partially addressed by methods like normalizing flows~\cite{nf-2015} and LVAE~\cite{lvae-2016}.

The proposed symmetric learning approach relaxes these restrictions and enables VAE learning when both the data and 
latent distributions are accessible only through sampling. This approach also applies to more complex models, such 
as Hierarchical VAEs (HVAEs), and leads to simpler algorithms for training. The experiments provided in the paper show that models obtained 
from this training approach are comparable to those achieved through ELBO learning.

In the standard VAE framework, we train the encoder  and decoder through maximizing the ELBO objective, i.e.
given the true underlying distribution of data $\pi(\boldsymbol{x}), \boldsymbol{x} \in \mathcal{X}$ and underlying
the distribution in latent variable $\pi(\boldsymbol{z}), \boldsymbol{z} \in \mathcal{Z}$, we maximize ELBO:
$$
\mathcal{L}_{B} = \mathrm{ELBO} = \EX_{\pi(\boldsymbol{x})} \bigl[ \EX_{q_{\boldsymbol{\phi}}(\boldsymbol{z} \mid \boldsymbol{x})} 
\log p_{\boldsymbol{\theta}}(\boldsymbol{x} \mid \boldsymbol{z}) - \mathrm{KL} (q_{\boldsymbol{\phi}}(\boldsymbol{z} \mid \boldsymbol{x}) || 
\pi(\boldsymbol{z})) \bigr]
$$
in order to obtain the pair of encoder $q_{\boldsymbol{\phi}}(\boldsymbol{z} \mid \boldsymbol{x})$ and decoder
$p_{\boldsymbol{\theta}}(\boldsymbol{x} \mid \boldsymbol{z})$. It is necessary to define the model distribution $p(\boldsymbol{z})$ in 
closed form to keep the computation of KL divergence tractable. This is an issue in case the $\pi(\boldsymbol{z})$ is complex, and we cannot model it 
by a simple distribution family.  Another necessity is that the $q_{\boldsymbol{\phi}}(\boldsymbol{z} \mid \boldsymbol{x})$ allow the 
\textit{reparametrization trick}. 

The authors propose a new algorithm for learning the encoder $q_{\boldsymbol{\phi}}(\boldsymbol{z} \mid \boldsymbol{x})$ and 
decoder $p_{\boldsymbol{\theta}}(\boldsymbol{x} \mid \boldsymbol{z})$ in case of \textit{semi-supervised} and 
\textit{unsupervised} learning:
\begin{itemize}
    \item \textit{Semi-supervised learning}: We can draw i.i.d samples from underlying unknown distributions $(\boldsymbol{x},\boldsymbol{z})  \sim \pi(\boldsymbol{x},\boldsymbol{z})$
    and its marginals: $\boldsymbol{x} \sim \pi(\boldsymbol{x})$, $\boldsymbol{z} \sim \pi(\boldsymbol{z})$.
    \item \textit{Unsupervised learning}: We can draw only $\boldsymbol{x} \sim \pi(\boldsymbol{x})$. The latent space is modelled
    through the choice of model $p(\boldsymbol{z})$.
\end{itemize}

The encoder and decoder belong to the exponential family and allow for tractable computation of log density and its derivatives.
\begin{align*}
    p_{\boldsymbol{\theta}}(\boldsymbol{x} \mid \boldsymbol{z}) &\propto \exp \bigl[ \boldsymbol{\Theta}(\boldsymbol{x}) \cdot \boldsymbol{f}_{\boldsymbol{\theta}}(\boldsymbol{z}) \bigr] \\
    p_{\boldsymbol{\phi}}(\boldsymbol{x} \mid \boldsymbol{z}) &\propto \exp \bigl[ \boldsymbol{\Phi}(\boldsymbol{z}) \cdot \boldsymbol{g}_{\boldsymbol{\phi}}(\boldsymbol{x}) \bigr]
\end{align*} 
where $\boldsymbol{\Theta}(\boldsymbol{x}) \in \mathbb{R}^n$ and $ \boldsymbol{\Phi}(\boldsymbol{z}) \in \mathbb{R}^m$ are sufficient
statistics. The variables $\boldsymbol{x}$ and $\boldsymbol{z}$ can be either discrete or continuous depending on the choice of an exponential family (e.g. Bernoulli or Gaussian).

The authors provide a new optimization function, which is motivated by finding a \textit{Nash equilibrium} for a two-player game 
where players' strategies are represented through the encoder and decoder distributions, respectively and the utility function is
a sum of the player expectation w.r.t his strategy~\cite{sym-learning-2023}. 
The objectives are 
\begin{align*}
    \mathcal{L}_{p}(\boldsymbol{\theta},\boldsymbol{\phi}) &= \EX_{\pi(\boldsymbol{x},\boldsymbol{z})}\bigl[\log p_{\boldsymbol{\theta}}(\boldsymbol{x},\boldsymbol{z})\bigr]+
                    \EX_{\pi(\boldsymbol{z})}\bigl[\log p_{\boldsymbol{\theta}}(\boldsymbol{z})\bigr] +
                    \EX_{\pi(\boldsymbol{x})}\EX_{q_{\boldsymbol{\phi}}(\boldsymbol{z} \mid\boldsymbol{x})}\bigl[\log p_{\boldsymbol{\theta}}(\boldsymbol{x},\boldsymbol{z})\bigr]\\
    \mathcal{L}_{q}(\boldsymbol{\theta},\boldsymbol{\phi}) &= \EX_{\pi(\boldsymbol{x},\boldsymbol{z})}\bigl[\log q_{\boldsymbol{\phi}}(\boldsymbol{z} \mid \boldsymbol{x})\bigr] + 
    \EX_{\pi(\boldsymbol{z})}\EX_{p_{\boldsymbol{\theta}}(\boldsymbol{x} \mid\boldsymbol{z})} \bigl[\log q_{\boldsymbol{\phi}}(\boldsymbol{z} \mid \boldsymbol{x})\bigr]
\end{align*}
for semi-supervised training and 
\begin{align}
    \mathcal{L}_{p}(\boldsymbol{\theta},\boldsymbol{\phi}) &= \EX_{\pi(\boldsymbol{x})}\EX_{q_{\boldsymbol{\phi}}(\boldsymbol{z} \mid\boldsymbol{x})}\bigl[\log p_{\boldsymbol{\theta}}(\boldsymbol{x},\boldsymbol{z})\bigr] \label{eq:unsup_objective_theta}\\
    \mathcal{L}_{q}(\boldsymbol{\theta},\boldsymbol{\phi}) &= \EX_{p_{\boldsymbol{\theta}}(\boldsymbol{x},\boldsymbol{z})} \bigl[\log q_{\boldsymbol{\phi}}(\boldsymbol{z} \mid \boldsymbol{x})\bigr] \label{eq:unsup_objective_phi}
\end{align}
for unsupervised training with the following interpretation: We maximize the decoder and encoder likelihood of the training data simultaneously. 
The mixed terms reinforce the encoder-decoder consistency. This corresponds to the maximization of the ELBO objective since we can 
rewrite the ELBO into:
$$
\EX_{\pi(\boldsymbol{x})} \bigl[ 
\log p_{\boldsymbol{\theta}}(\boldsymbol{x}) - \mathrm{KL} (q_{\boldsymbol{\phi}}(\boldsymbol{z} \mid \boldsymbol{x}) || 
p_{\boldsymbol{\theta}}(\boldsymbol{z} \mid \boldsymbol{x})) \bigr]
$$
After inspecting the terms, we see that the ELBO goal is the same as above: To maximize the data likelihood and reinforce the consistency of the decoder-encoder pair 
simultaneously. 
\subsection{Hiearchical VAEs}
The algorithm can also be adopted for hierarchical VAE. Let us assume that we have HVAE with $M+1$ layers, i.e. $\boldsymbol{z}$ consists of 
$\boldsymbol{z}_0,\boldsymbol{z}_1,\dots,\boldsymbol{z}_m$, where the encoder models correspond to the LVAE, and we can sample $\boldsymbol{x} \sim \pi(\boldsymbol{x})$.
The encoder and decoder factorize (the ordering is in reverse to the one in eq.~\ref{eq:hvae_prior} and eq.~\ref{eq:hvae_posterior}):
\begin{align*}
    p_{\boldsymbol{\theta}}(\boldsymbol{x},\boldsymbol{z})&=p_{\boldsymbol{\theta}}(\boldsymbol{z}_{0}) \prod_{t=1}^{M}\bigl[p_{\boldsymbol{\theta}}(\boldsymbol{z}_{t}\mid\boldsymbol{z}_{<t}) \bigr]  p_{\boldsymbol{\theta}}(\boldsymbol{x}\mid\boldsymbol{z})  \\
    q_{\boldsymbol{\phi}}(\boldsymbol{x},\boldsymbol{z}) &=\pi(\boldsymbol{x})q_{\boldsymbol{\phi}}(\boldsymbol{z}_{0}\mid\boldsymbol{x}) \prod_{t=1}^{M}  q_{\boldsymbol{\phi}}(\boldsymbol{z}_{t}\mid\boldsymbol{z}_{<t},\boldsymbol{x}) 
\end{align*}
where the encoder shares the parameter as described in~\ref{eq:lvae_encoder}, the objectives remain as in unsupervised case (eq.~\ref{eq:unsup_objective_theta} and eq.~\ref{eq:unsup_objective_phi}).
The terms can be decomposed due to the factorization of decoder and encoder and are thus tractable. If there is 
access to the samples $(\boldsymbol{x},\boldsymbol{z}_0) \sim \pi(\boldsymbol{x},\boldsymbol{z}_0)$, e.g. segmentation task with target masks, we can utilize them 
by adding terms
$$
\EX_{\pi(\boldsymbol{x},\boldsymbol{z}_0)}\EX_{q(\boldsymbol{z}_{>0} \mid\boldsymbol{z}_0,\boldsymbol{x})} \log p_{\_\theta}(\boldsymbol{x},\boldsymbol{z}) 
\quad\text{and}\quad
\EX_{\pi(\boldsymbol{x},\boldsymbol{z}_0)} \log q_{\_\phi}(\boldsymbol{z}_0\mid\boldsymbol{x})
$$
to the decoder and encoder, respectively. 
