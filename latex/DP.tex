% arara: pdflatex: { synctex: yes }
% arara: makeindex: { style: ctuthesis }
% arara: bibtex

% The class takes all the key=value arguments that \ctusetup does,
% and a couple more: draft and oneside
\documentclass[twoside]{ctuthesis}

\ctusetup{
	preprint = \ctuverlog,
	mainlanguage = english,
	titlelanguage = english,
%	mainlanguage = czech,
	otherlanguages = {czech},
	title-czech = {Částečné učení s učitelem pro časově-prostorovou segmentaci satelitních snímků},
	title-english = {Semi-Supervised Learning for Spatio-Temporal Segmentation of Satellite Images},
	%subtitle-czech = {Cesta do tajů kdovíčeho},
	%subtitle-english = {Journey to the who-knows-what wondeland},
	doctype = M,
	faculty = F3,
 	department-czech = {Katedra kybernetika},
	department-english = {Department of Cybernetics},
	author = {Antonín Hruška},
	supervisor = {doc. Boris Flach, Dr. rer. nat. habil.},
	%supervisor-address = {ss},
	%supervisor-specialist = {John Doe},
	fieldofstudy-english = {Cybernetics and Robotics},
	%subfieldofstudy-english = {Mathematical Modelling},
	fieldofstudy-czech = {Kybernetika a robotika},
	%subfieldofstudy-czech = {Matematické modelování},
	keywords-czech = {Segmentace obrazu, SSL,HVAE,LVAE,symmetric learning, MixMatch,CityScape},
	keywords-english = {Image segmentation, SSL, VAE,HVAE, LVAE, symetric learning, MixMatch, CityScape},
	day = 26,
	month = 5,
	year = 2023,
	specification-file = {Thesis_Assignment_Antonin_Hruska_Semi-Supervised_Learning_for_Spatio-Temporal_Segmentation_of_Satellite_Images},
	front-specification = true,
	front-list-of-figures = true,
	front-list-of-tables = true,
    pkg-hyperref = true,
% Does not work	pkg-biblatex = true
%	monochrome = true,
%	layout-short = true,
}

\ctuprocess

\addto\ctucaptionsczech{%
	\def\supervisorname{Vedoucí}%
	\def\subfieldofstudyname{Studijní program}%
}

\ctutemplateset{maketitle twocolumn default}{
	\begin{twocolumnfrontmatterpage}
		\ctutemplate{twocolumn.thanks}
		\ctutemplate{twocolumn.declaration}
		\ctutemplate{twocolumn.abstract.in.titlelanguage}
		\ctutemplate{twocolumn.abstract.in.secondlanguage}
		\ctutemplate{twocolumn.tableofcontents}
		\ctutemplate{twocolumn.listoffigures}
	\end{twocolumnfrontmatterpage}
}

% Theorem declarations, this is the reasonable default, anybody can do what they wish.
% If you prefer theorems in italics rather than slanted, use \theoremstyle{plainit}
\theoremstyle{plain}
\newtheorem{theorem}{Theorem}[chapter]
\newtheorem{corollary}[theorem]{Corollary}
\newtheorem{lemma}[theorem]{Lemma}
\newtheorem{proposition}[theorem]{Proposition}

\theoremstyle{definition}
\newtheorem{definition}[theorem]{Definition}
\newtheorem{example}[theorem]{Example}
\newtheorem{conjecture}[theorem]{Conjecture}

\theoremstyle{note}
\newtheorem*{remark*}{Remark}
\newtheorem{remark}[theorem]{Remark}

\setlength{\parskip}{4ex plus 0.2ex minus 1ex}

% Abstract in Czech
\begin{abstract-czech}
\todo{}
\end{abstract-czech}

% Abstract in English
\begin{abstract-english}
 \todo{}
\end{abstract-english}

% Acknowledgements / Podekovani
\begin{thanks}
	I would like to express my deepest gratitude to Doc. Boris Flach, Dr. rer. nat. habil., without 
	whom this endeavor would not have been possible. I am immensely thankful to him for the professional 
	guidance he provided me with, as well as his kind words and moral support during the challenging times. 
	I would also like to extend my heartfelt thanks to all the teachers, lecturers, and individuals who generously
	 dedicated their time, effort, and passion to teach me new things and propel me forward. 

	 I am also grateful for the access to the computational infrastructure of the OP VVV funded project CZ.$02.1.01/0.0/0.0/16\_019/0000765$. 
	 The support provided by the "Research Center for Informatics" through this project is sincerely acknowledged and appreciated.

	 In addition, I would like to express my gratitude to my family, partner, and close friends who have been unwavering 
	 in their support throughout my studies and have played a significant role in helping me complete this thesis. Their
	encouragement and assistance have been invaluable to me, and I am truly grateful for their presence in my life.
\end{thanks}

% Declaration / Prohlaseni
\begin{declaration}
\label{sec:decleration}
I declare that the presented work was developed independently and that I have listed all sources of information
used within it in accordance with the methodical instructions for observing the ethical principles in the preparation 
of university theses.

In particular, I declare that I have not used natural language processing models to generate the ideas and themes used in the thesis. 
Nevertheless, I have used such tools (Grammarly, DeepL) to correct any grammatical or syntactical errors or to translate the original 
Czech version of the text into English and to improve the wording of such text. I have also used the Chat-GPT models for creating code
snippets in both Python and Latex, as well as having a second reader of the thesis. 

In Prague, 26th May, 2023
\end{declaration}

\usepackage{diagbox}
\usepackage{siunitx}
\sisetup{output-decimal-marker={.},exponent-product=\cdot}
\usepackage{amsmath}
\usepackage{amsfonts}
\usepackage{amssymb}
\usepackage{graphicx}
\usepackage{tikz}
\usetikzlibrary{calc}
\usetikzlibrary{shapes, arrows.meta, positioning}
\usetikzlibrary{positioning,automata}
\usepackage{subcaption}
\usepackage{xcolor}
\usepackage{csquotes}
\usepackage{makecell}
\usepackage{physics}
\usepackage{lipsum}
\usepackage{listings}
%\usepackage{julia-mono-listings}

\usepackage{algorithm}
\usepackage{algpseudocode}

% CUSTOM FONT SETTING
\usepackage{tgpagella}

\usepackage{xkeyval}	% Inline todonotes
\usepackage[textsize = footnotesize]{todonotes}
\presetkeys{todonotes}{inline}{}

\usepackage[style=numeric]{biblatex}
\addbibresource{Diploma Thesis.bib}
\DefineBibliographyStrings{english}{%
  bibliography = {References},
}

\usepackage{url}
\usepackage{tabularx}
\usepackage{bookmark}

%%% --- The following two lines are what needs to be added --- %%%
\setcounter{biburllcpenalty}{7000}
\setcounter{biburlucpenalty}{8000}

% Definition of new commands
\newcommand{\argmax}{\mathop{\rm argmax}}
\newcommand{\argmin}{\mathop{\rm argmin}}
\newcommand{\arctg}{\mathop{\rm arctg}}
\newcommand{\tg}{\mathop{\rm tg}}
\newcommand{\aff}{\mathop{\rm aff}}
\newcommand{\conv}{\mathop{\rm conv}}
%\newcommand{\rank}{\mathop{\rm rank}}
\newcommand{\diag}{\mathop{\rm diag}}
\newcommand{\sgn}{\mathop{\rm sgn}}
\newcommand{\Null}{\mathop{\rm null}}
\newcommand{\Rng}{\mathop{\rm rng}}
\newcommand{\dist}{\mathop{\rm dist}}
\renewcommand{\d}[1]{\mbox{\rm d}#1}

%\renewcommand{\phi}{\varphi}

% Change of symbols for footnotes
\renewcommand{\thefootnote}{\fnsymbol{footnote}}
\graphicspath{{./figs/}}
% Boldface vectors. Example: \_x
\def\_#1{\protect\mathchoice
    {\mbox{\boldmath $\displaystyle\bf#1$}}
    {\mbox{\boldmath $\textstyle\bf#1$}}
    {\mbox{\boldmath $\scriptstyle\bf#1$}}
    {\mbox{\boldmath $\scriptscriptstyle\bf#1$}}}

% Blackboard letters. Example: \bb R
\def\bb#1{\mathbb{#1}}

% Custom commands
\def\maxjerk{j_{\mathrm{M}}}

% Rozsireni tabulky kvuli nahnacani symbolu.
\renewcommand{\arraystretch}{1.25}

% Cpp, C#
\newcommand{\CC}{C\nolinebreak\hspace{-.05em}\raisebox{.4ex}{\tiny\bf +}\nolinebreak\hspace{-.10em}\raisebox{.4ex}{\tiny\bf +}}
\newcommand{\CS}{C\nolinebreak\hspace{-.05em}\raisebox{.6ex}{\tiny\bf \#}}

% repeating footnote
\newcommand{\savefootnote}[2]{\footnote{\label{#1}#2}}
\newcommand{\repeatfootnote}[1]{\textsuperscript{\ref{#1}}}

% Colour definitions
\definecolor{myBlue}{HTML}{6666FF}
\definecolor{myGreen}{HTML}{66B366}
\definecolor{myRed}{HTML}{FF6666}

% Make sumint symbol


\DeclareRobustCommand{\bbone}{\text{\usefont{U}{bbold}{m}{n}1}}

\DeclareMathOperator{\EX}{\mathbb{E}}

\begin{document}

\maketitle


\chapter{Introduction}
% Background (1-2 pages)

% Introduction to the topic and its importance
% Explanation of the real-world problem or scenario that the thesis aims to address
% Overview of the current state of research and any recent advances in the field
% Discussion of any challenges or limitations of existing approaches to the problem or scenario

Satellite imagery provides a wealth of information about our planet and has become a standard tool for monitoring, predicting and understanding 
the change in vegetation, agriculture and human environmental impact. With advancements in satellite technology, it is now possible to collect
considerably large amounts of high-resolution multispectral images over time, enabling researchers to perform complex spatiotemporal analyses of these 
datasets. However, the magnitude of the raw data makes it challenging to process and analyze effectively. Moreover the remote sensing 
still faces other challenges, which are rare in other areas of computer vision. Those are among others partial measurements 
(i.e. cloud cover, electro-magnetic (EM) interference), geolocation, different quality of measurement (spatial resolution, different EM bands) and 
calibration (issue of atmospheric reflectance).

From a machine learning perspective, the main challenges are the quite large amounts of unannotated data and partially missing measurements.
To adress the first, semi-supervised learning \cite{ssl-book-2006} (SSL) has shown promise in leveraging the abundance of unlabeled
data to improve model performance. SSL algorithms aim to learn from labeled and unlabeled data, using the labeled data to guide the 
learning process and the unlabeled data to regularize the model. Regarding the issue of partially missing measurements, generative
models, such as variational autoencoders (VAEs) \cite{intro-vae-2019}, have shown the potential to fill in missing data gaps. 
By learning a generative model of the data distribution, VAEs can be used to impute missing values in a dataset, making it possible to 
fully utilize the available data.

The thesis topic is motivated by the real-world problem of segmenting national park's satellite imagery to monitor and predict 
its forest development. The forest development prediction could enable national park rangers to respond proactively to protect forest vegetation and thus
improve the national park protection. The thesis aims to create an approach that could be used in preprocessing to obtain segmentation for many images and 
generate inpainting for partially missing measurements. The project is funded by European Space Agency (ESA)
\footnote[1]{\url{https://eo4society.esa.int/projects/spatiotemporal-sen2vhr/}}.

The authors of \cite{sym-learning-2023} proposed a novel SSL algorithm that employs symmetrical learning with hierarchical VAEs, which is capable of
handling exponential families of distribution, not just multivariate Gaussians for the latent variable. We adapt this algorithm to the 
land-cover segmentation task, where the learning process from only partially available inputs is necessary. This is due to missing measurements 
which are mainly caused by cloud or snow cover. 
Our approach involves combining a reasonably sized model inspired by U-net architecture and its skip connections\cite{unet-2015} with hierarchical VAEs, enabling
us to obtain a latent space distribution that represents the desired segmentation. Additionally, this setup will allow us to generate both segmentation from images 
and images from segmentation.

We will compare our approach with MixMatch \cite{mixmatch-2019} algorithm, which unifies the consistency regularization methods with proxy-labeling
methods \cite{ssl-overview-2020} and is the cornerstone for other algorithms such as ReMixMatch \cite{remixmatch-2020} and FixMatch \cite{fixmatch-2020}.
To our knowledge, the MixMatch algorithm has only been used for classification tasks (CIFAR10, SVHN). We will adapt it, so that is applicable
to the segmentation task. To verify its suitability and compare it with our method, we will use CityScape dataset and standard metrics used in 
segmentation, such as intersection over union (IoU), plain accuracy  and others. We will also validate the generative capabilities of the novel approach (e.g. 
inpainting of missing data).
\todo{Rewrite it once the actual evaluation is done.}

In chapter \ref{motivation-theory} we provide a brief introduction to SSL and its methods.  We will discuss the main ideas 
and cornerstones of MixMatch and the novel symmetrical equilibrium learning, namely consistency regularization with proxy 
labelling methods and VAE framework respectively. 
We also introduce the hieachical models (HVAE) and ladder variational autoencoders (LVAE) in this chapter as a necessary foundation for 
the novel algorithm. The description of the segmentation task, dataset, metrics and models architecture is provided
in chapter \ref{methods}. The experiments and results are available in \ref{results} with a summary of main findings and contributions in \ref{conclusions}. 
We make all code publicly available \footnote[2]{At \url{https://github.com/hruskan1/SSL-diploma-thesis}}. 




\chapter{State of the art of SSL}
\label{motivation-theory}
In this chapter, we will present a brief introduction to semi-supervised learning in section \ref{sec:ssl-introduction}. 
We will discuss the key concepts and principles underlying semi-supervised learning and its significance in machine learning. 
Furthermore, we will classify the algorithms used in semi-supervised learning into various groups based on the different 
ideas and paradigms they employ.

Additionally, we will provide a more detailed overview of MixMatch and Variational Autoencoders in sections \ref{sec:mix-match} and \ref{sec:vae}, with
a particular focus on Variational Autoencoders and their variants as they are cornerstone of the novel approach proposed by \cite{sym-learning-2023}, which 
we finally introduce section \ref{sec:symmetrical-learning}.
\todo{Add reference to unpublished paper}

% SotA SSL
\section{SSL introduction}
\label{sec:ssl-introduction}
Semi-Supervised Learning (SSL) is an essential subfield of Machine Learning (ML) that aims to improve model performance by leveraging 
both labeled and unlabeled data. In many real-world scenarios, obtaining labeled data is expensive and time-consuming, whereas 
unlabeled data is abundant and relatively easy to acquire. Therefore, SSL algorithms seek to learn from both labeled and unlabeled 
data to improve model generalization and achieve higher accuracy. Unlike supervised learning, where models rely entirely on labeled 
data, SSL algorithms use a small amount of labeled data to guide the model's learning process while exploiting the vast amounts of 
unlabeled data to extract useful features and improve its predictions. In recent years, there has been a growing interest in developing
novel SSL algorithms that can tackle complex problems and achieve state-of-the-art performance, making SSL a rapidly evolving field of 
research. 

Typically the training dataset $\mathcal{D}$ can be divided into two subsets $\mathcal{D} = \mathcal{D}_l \cup \mathcal{D}_u$:
\begin{align*}
    \mathcal{D}_l = \{(\mathbf{x}_1,y_1),\dots, (\mathbf{x}_l,y_l)\},\quad \mathcal{D}_u = \{(\mathbf{x}_{l+1}),\dots,(\mathbf{x}_{u})\},
\end{align*}
In this \textit{standard} setting, SSL can be viewed as supervised learning, where 
the \textit{unlabeled} data provide additional information on the underlying distribution of the examples $\mathbf{x}$.
We will refer to this setting in this thesis, however there are also different formulations of the SSL problem, such as \textit{SSL with constraints}
and others~\cite[p. 1]{ssl-book-2006}.

\begin{quote}
    \textit{
    ``A natural question arises: is semi-supervised learning meaningful? More precisely:
    in comparison with a supervised algorithm that uses only labeled data, can one
    hope to have a more accurate prediction by taking into account the unlabeled
    points? \dots Yes, however there is an important prerequisite: that the
    distribution of examples, which the unlabeled data will help elucidate, be relevant
    for the classification problem.
    \dots One should thus not be too surprised that for semi-supervised learning to work,
    certain assumptions will have to hold.}''~--~Chappele~et~\textit{al}~\cite[p. 4]{ssl-book-2006}
\end{quote}

\subsection{Assumptions in SSL}
As stated in the quote above, several assumptions are necessary for SSL algorithms to work~\cite[p. 5]{ssl-book-2006}. 
Some of them are well known from unsupervised learning:
\begin{itemize}
    \item \textbf{The Smoothness Assumption}: \textit{If two points $\mathbf{x}_1$ and $\mathbf{x}_2$ lies nearby in high-density region, 
    then the desired outputs $\mathbf{y}_1$ and $\mathbf{y}_2$ should be similar.} This assumption generalizes the supervised learning assumption,
    where the same holds if $\mathbf{x}_1$ and $\mathbf{x}_2$ \textit{are close} (not necessarily in high-density region). Due to transitivity, 
    the assumption clusters the data into high-density clusters, and many clusters can share the same output value.
    \item \textbf{The Cluster Assumption}: \textit{Points in one cluster are likely to be of the same class}, or in other words, \textit{the decision 
    boundary should be located in low-denisty region.} This assumption is a special case of the previously mentioned assumption, as clusters are often 
    considered regions with a high density of data. However, it is presented independently as it is easier to understand and has motivated several 
    unsupervised algorithms such as K-means and others. %\cite{k-means-1967,k-means-1982}
    \item \textbf{The Manifold Assumption}: \textit{The data lie along low-dimensional latent manifolds inside that high-dimensional space.} This
    assumption tries to overcome the \textit{curse of dimensionality}. Simply put, as the dimension grows, the sparsity of data increases, which makes 
    clustering impossible, as there are no clusters to be found. If the manifold assumption holds, we can search for a mapping into such a 
    low-dimensional manifold, in which clustering is possible. There are several unsupervised algorithms that utilize this assumption, such as PCA, 
    MDS, ISOMAP, and t-SNE.
\end{itemize}
\todo{Should I cite the algorithms?}

% Another important perspective on the proposed question is Vapnik's principle\cite[p. 7]{ssl-book-2006}:

% \begin{quote}
%     ``When solving a problem of interest, do not solve a more general problem as an intermediate step. 
%     Try to get the answer you really need, not a more general one''~--~Vapnik \cite{vapnik-quote-2006}
% \end{quote}

% We call the method \textit{transductive} if it only makes predictions for the test points (assuming the test set is available during the learning process). This is 
% in contradiction to \textit{inductive} learning, where the goal is to infere a prediction function defined on the whole set $\mathcal{X}$. In the idea of Vapnik's
% quote, the \textit{transductive} learning is more direct then \textit{inductive}.

% Suppose there is a transductive algorithm which result outperforms the one produced by an inductive algorithm trained on the same labeled data (but discarding the 
% unlabeled data). Then the performance difference could be due to one of the following (or a combination of them)
% \begin{enumerate}
%     \item transduction follows Vapnik's principle more closely than induction does, or
%     \item the transductive algorithm takes advantage of the unlabeled data in a way similar to semi-supervised learning algorithms. 
% \end{enumerate}
% There is a lot of evidence for the latter with no empirical results that would selectively support the first point\cite[p. 7]{ssl-book-2006}. However the 
% insightful discussion with different viewpoints is available in \cite[chap. 25]{ssl-book-2006} and the idea of the transduction and its effect on SSL is still open.

\subsection{SSL methods}
SSL algorithms can be categorized into these following groups based on their motivation, making it easier to navigate and understand them~\cite{ssl-overview-2020}:
\begin{itemize}
        \item \textbf{Consistency Regularization}: According to the smoothness assumption, if the input $\mathbf{x}$ and its perturbed version $\tilde{\mathbf{x}}$, 
        are close to each other, their corresponding outputs, $\mathbf{y}$ and $\tilde{\mathbf{y}}$, should also be similar. By minimising the 
        distance between the model outputs$f_\theta(\mathbf{x})$ and $f_\theta(\tilde{\mathbf{x}})$, where the distance can be measured using 
        a variety of techniques, such as mean square error (MSE) or Kullback-Leibler (KL) divergence, we can train the model to make consistent
        predictions on both the original and perturbed inputs~\cite{temporal-ensembling-2017,regularization-&-pertrubations-2016}.
        We can also use other divergence techniques, such as Jeffrys divergence or Jensen-Shanon (JSD) divergence, which have the advantage of 
        being symmetric with respect to the inputs.  This leads to an expanded loss objective, where a new term is introduced for consistency 
        regularization:
        $$
        \mathcal{L} = \sum_{\mathbf{x},y \in \mathcal{D}_l}l(\mathbf{x},y) + \sum_{\mathbf{x} \in \mathcal{D}_u} d(f_\theta(\mathbf{x}),f_\theta(\tilde{\mathbf{x}}))
        $$
        where $l(\mathbf{x},y)$ corresponds to the standard supervised loss for given task and $d(\cdot,\cdot)$ corresponds to the one of the mentioned metrics. 

    \item \textbf{Proxy-label Methods}: These methods are based on an (iterative) scheme, where the model generates the proxy label on unlabeled data (or parts 
        thereof) using the prediction function itself or some variant of it~\cite{psuedo-label-2013}. These labels are then taken as targets for the next iteration.
        Although the proxy labels are often and/or weak, the methods can provide additional information for training. These methods can be divided into two groups: 
        self-training, where the model produces the proxy label itself, and \textit{multi-view learning}, where the proxy labels are produced by (multiple) models
        trained on different views of the training data. The idea of multi-view learning is exactly the same as bootstrapping.

    \todo{should cite bootstraping?}

    \item \textbf{Generative Models}:\label{generative-modelling} The \textit{generative} models try to model the feature density $p(\mathbf{x})$ or even joint density 
        $p(\mathbf{x},y)$ by some unsupervised learning procedure (i.e. maximum likelihood estimation (MLE)). An inference can be then obtained by Bayes 
        inference rule (for a given loss $l$):
        \begin{equation*}
            f^{\star}(\mathbf{x}) = \argmin_{y^\prime\in \mathcal{Y}} \sum_{y \in \mathcal{Y}}(y|\mathbf{x})l(y,y^\prime)
        \end{equation*}
        where conditional probability $p(y|x)$ can be obtained through Bayes theorem:
        \begin{equation*}
            p(y|\mathbf{x}) = \frac{p(\mathbf{x},y)}{p(\mathbf{x})} =\frac{p(\mathbf{x}|y)p(y)}{\int_\mathcal{Y} p(\mathbf{x}|y)p(y) \d y}
        \end{equation*}

        After training a model, we can use it to generate new samples from a \textit{modelled} distribution $p_\mathbf{\theta}(\mathbf{x})$ at any time. 
        This allows us to obtain features that were not present in the original training set, but the quality of these new features depends on
        how closely our model approximates the true underlying distribution $p^\star(\mathbf{x})$ represented by the training set distribution 
        $p_{\mathcal{D}}(\mathbf{x})$, which is also known as the \textit{evidence}. Therefore, the quality of the generated samples depends on the 
        accuracy of the model's approximation to the true distribution.

        Generative models are used in SSL because they can easily incorporate the unlabeled data points (compared to \textit{discriminative} models, which only focus on estimating
        $p(y|\mathbf{x})$ and cannot directly exploit the infromation in $p(\mathbf{x})$). On the other hand, the \textit{discriminative} models fulfill the Vapnik's principle and in its
        sense can provide comparable results even without the use of the unlabelled data.  In a broader context, SSL can be viewed in the field of generative models as 
        either classification with supplementary information on the marginal density or unsupervised clustering with additional information, i.e., labels of a subset 
        of points. A reasonable requirement on SSL would be that any valid SSL technique should surpass baseline methods by a significant margin in a range of across a
        variety of practical and relevant scenarios. 
    \item \textbf{Graph-Based Methods}:
    Semi-supervised methods that are based on graphs establish a graph structure where the labeled and unlabeled examples in the dataset constitute the nodes, and the 
    similarity between examples is reflected by edges that may be weighted. Typically, these methods presume label smoothness throughout the graph. Graph-based 
    approaches are characterized as nonparametric, discriminative, and transductive in nature~\cite{another-survey-2008}.
\end{itemize}

When talking about \textit{consistency regularization}, one should mention also \textbf{Entropy minimization}~\cite{entropy-min-2004} as it shares the same underlying concept
of \textit{smoothness assumption} and aims at same result: Moving the decision boundary into low-density region. The entropy minimization encourage the 
network to make confident (i.e. low-entropy) predictions on unlabled data regardless of the predicted class and thus moving the decision boundary away from any point in dataset.
This can be obtained by adding an entropy minimization term:
\begin{equation*}
    H(p) = -\sum_{k=1}^{C} p_\mathbf{\theta}(y|\mathbf{x})_k \log p_\mathbf{\theta}(y|\mathbf{x})_k
\end{equation*}
Nevertheless, the neural networks (NN) can quickly overfit to low confident points early on in the learning process. This is caused by their high capacity~\cite{how-to-evalute-ssl-2018}.
The Entropy minimization on its now does not lead to strong results, however it is often combined with different approaches to improve their performance~\cite{ssl-overview-2020}.

\section{MixMatch}
\label{sec:mix-match}
We have selected the MixMatch algorithm as a reference algorithm for the comparison as it yielded state of the art results. This \textit{holistic} approach
was proposed by David Berthelot et \textit{al.} in 2019~\cite{mixmatch-2019} and combines several ideas and components from classical dominant paradigms of SSL.
It is the cornerstone for a whole branch of new algorithms such as ReMixMatch~\cite{remixmatch-2020} and FixMatch~\cite{fixmatch-2020}. Namely, it combines 
\textit{consistency regularization} and \textit{proxy-labeling} with \textit{entropy minimization}.It also utilizes other forms of regularizations, such as
\textit{data augmentation}, \textit{exponentially weighted average of network weights}~\cite{mean-teacher-2018},\textit{weight decay}~\cite{weight-decay-2019} and 
\textit{MixUp} procedure~\cite{mixup-2018}. The consistency regularization is obtained through loss term, the proxy-labeling occures in stage of label 
guessing ~(\ref{label-guessing}) and the entropy minimization is applied in form of sharpening procedure~(\ref{sharpening}).

The algorithm itself is composed of several steps and provides augmented inputs to the model with \textit{guessed} labels. The batched augmented inputs are propagated 
through the network and the standard semi-supervised loss containing the supervised and unsupervised term is computed from outputs of the model and the (guessed) 
labels. The gradient is backpropagated to the network's weights, meaning the MixMatch is applicable in the setting of Deep Learning (DL).
% todo change following x,u occurences. decide what to do with p.
Assume we have batch of labeled inputs $\mathcal{X}$ (with labels encoded as one-hot vectors with $L$ possible classes) and batch of unlabeled inputs $\mathcal{U}$ 
(without labels), both with same number of examples $n$. The SSL loss is defined as:
\begin{align*}
    \mathcal{X}^\prime,\mathcal{U}^\prime &= \text{MixMatch}(\mathcal{X}, \mathcal{U}, T, K, \alpha)\\
    \mathcal{L}_{\mathcal{X}} &= \frac{1}{|\mathcal{X}^\prime|}\sum_{x^\prime,p^\prime \in \mathcal{X}^\prime} H(p^\prime,f_\theta(x^\prime)) \\
    \mathcal{L}_{\mathcal{U}} &= \frac{1}{L|\mathcal{U}^\prime|}\sum_{u^\prime,q^\prime \in \mathcal{U}^\prime} ||q^\prime - f_\theta(u^\prime)||_2^2 \\
    \mathcal{L} &= \mathcal{L}_{\mathcal{X}} + \lambda_\mathcal{U} \mathcal{L}_{\mathcal{U}}
\end{align*}
where $H(p,q)$ is cross-entropy loss between distributions $p$ and $q$:
\begin{equation*}
    H(p,q) = -\sum_{k=1}^{C} p_k(x) \log q_k(x)
\end{equation*}
and $T,K,\alpha$ and $\lambda_\mathcal{U}$ are hyperparameters and $f_\theta(\cdot)$ represents the 
output of the model in the form of probability distribution. The $T$ is the \textit{temperature} in probability sharpening procedure, $K$ is the \textit{number of augmentations} applied to
unlabeled input $u$ and the $\alpha$ is the Beta distribution parameter for MixUp. The $\lambda_\mathcal{U}$ replaces the originaĺ normalizing factor and provides tuning knob for 
weighting the loss terms. 

\subsection{MixMatch algorithm}
The MixMatch algorithm consists of the following steps:
\begin{enumerate}
    \item \textbf{Data Augmentation}\label{data-augmentation}: Given the (stochastic) augmentation $A$, we transform each labeled features $x_i \in \mathcal{X}$ into $\tilde{x}_i$ while keeping the 
        original label $p$ unchanged. For unlabeled feature $u_j \in \mathcal{U}$, we produce $K$ augmented views $\tilde{u}_{j,k}$. Through this, we obtain $n$ labeled features and
        $nK$ unlabeled features. 
    \item \textbf{Label Guessing}\label{label-guessing}: For each of $K$ views of unlabeled feature $\tilde{u}_{j,k}$ we make the predictions with the current model 
        $\hat{q}_{j,k} = f_\theta(\tilde{u}_{j,k})$. We then compute the average
            \begin{equation*}
                \bar{q}_{j} = \frac{1}{K}\sum_{k=1}^K \hat{q}_{j,k}
            \end{equation*}
        for each unlabeled feature $u_j$.
    \item \textbf{Sharpening}\label{sharpening}: We sharpen the averaged prediction $\bar{q}_{j}$ to reduce its entropy through the operation:
            \begin{equation*}
                q_{j,c} = \text{Sharpen}(\bar{q}_{j},T)_c = \bar{q}_{j,c}^{\frac{1}{T}} \Big{/} \sum_{k=1}^{K} \bar{q}_{j,k}^{\frac{1}{T}} 
            \end{equation*}
        where $q_{j,c}$ corresponds to $c$-th element of vector $q_{j}$, representing the probability of $c$-th class. 
        The hyperparameter $T \in \mathbb{R}_{>0}$ is the \textit{temperature}. As $T\to 0$, the $\text{Sharpen}(p,T)$ approaches Dirac (one-hot) distribution, 
        therefore lowering the $T$ minimizes the entropy of $p$. We obtain the sharpened $q_{j}$ and we replicate it to each of $K$ views of feature $u_{j}$.
    \item \textbf{MixUp}: Before we continue in further description, we define the slightly alternated version of the vannila MixUp~\cite{mixup-2018}.
            For a pair of two features with their corresponding class probabilities $(x_1,p_1)$ and $(x_2,p_2)$, we define MixUp operation as following:
            \begin{align*}
                \lambda &\sim \text{Beta}(\alpha,\alpha) \\
                \lambda^\prime &= \max(\lambda,1-\lambda) \\
                x^\prime &= \lambda^\prime x_1 + (1-\lambda^\prime) x_2 \\
                p^\prime &= \lambda^\prime p_1 + (1-\lambda^\prime) p_2 \\
            \end{align*}
            where $\alpha$ is hyperparameter. Vannila MixUp omits the second equation (i.e. $\lambda^\prime = \lambda$), but it is crucial in MixMatch as you will
            see later. We define MixUp operation for (equally sized) sets\footnote{We should rather speak about sequences, as the sets do not have ordering. 
            Nevertheless,in the field of ML, we often neglect this difference. In reality, the computer memory always has the implicit ordering, which is used.} 
            as a MixUp per elements, i.e.
            \begin{equation*}
                \begin{split}
                    \text{MixUp}(\mathcal{D}_a,\mathcal{D}_b) = & \{\text{MixUp} \big{(} (x_{ai},y_{ai}),(x_{bi},y_{bi}) \big{)}\,|\,i \in {1,\dots,|\mathcal{D}_a|}  \}.
                \end{split}
            \end{equation*}
        Going back to MixMatch, the previous steps resulted in two batches with different sizes:
            \begin{align*}
                \mathcal{X}^\star &= \{ (\tilde{x}_i,p_i)\,|\, i \in \{1,\dots,n\}  \}, \, |\mathcal{X}^\star| = n \\
                \mathcal{U}^\star &= \{ (\tilde{u}_{j,k},q_j)\,|\, j \in \{1,\dots,n\},\,k \in \{1,\dots,K\}  \}, \, |\mathcal{U}^\star| = Kn 
            \end{align*}
        First we concatenate those two batches and shuffle them :
            \begin{align*}
                \mathcal{W} = \text{Shuffle}(\text{Concat}(\mathcal{X}^\star,\mathcal{U}^\star))
            \end{align*}
        we then slice the $\mathcal{W}$ into two: $\mathcal{W}_1$ of the same size as $\mathcal{X}^\star$ and $\mathcal{W}_2$ of the same size 
        as $\mathcal{U}^\star$ and we compute MixUp for both labeled and unlabeled sets:
        \begin{align*}
            \mathcal{X}^\prime &= \text{MixUp}(\mathcal{X}^\star,\mathcal{W}_1) \\
            \mathcal{U}^\prime &= \text{MixUp}(\mathcal{U}^\star,\mathcal{W}_2)
        \end{align*}
        The definition of $\lambda^\prime$ in alternated MixUp ensures, that the $(x^\prime,y^\prime)$ is always closer to the 
        $(x_1,y_1)$ then to $(x_2,y_2)$, which is important as it may happen, that the $\mathcal{W}_1$ will contain features from $\mathcal{U}$
        and we need to compute individual loss components appropriately. In other words, the $\mathcal{X}^\prime$ and $\mathcal{U}^\prime$ are always 
        closer to the $\mathcal{X}^\star$, resp. $\mathcal{U}^\star$ so the computed loss corresponds to the original inputs, i.e. batches $\mathcal{X}$, 
        resp. $\mathcal{U}$.
\end{enumerate}
\begin{figure}[t]
    \centering
    \includegraphics[width=\textwidth]{mixmatch_label_guessing.png}
    \caption[Mixmatch label guessing]{Data augmentation, label guessing and sharpening procedure visualized for unlabeled datapoint. The unlabeled image is at first $K$ 
    times augmented, each augmentation is then classified by the current model. The predictions are then averaged and sharpened. 
    Source~\cite{mixmatch-2019}}
    \label{fig:mixmatch}
\end{figure}

\begin{algorithm}[H]
 \caption{MixMatch}
 \label{alg:mixmatch}
 \begin{algorithmic}[1]
   \State \textbf{Input:} Batch of labeled examples and their one-hot labels $X = ((x_i, p_i);\,i \in (1, \dots, n))$, batch of unlabeled examples $U = (u_i; i \in (1, \dots, n))$, sharpening temperature $T$, number of augmentations $K$, Beta distribution parameter $\alpha$ for MixUp.
   \For{$i = 1$ \textbf{to} $n$}
    \State $\bar{x}_i = \text{Augment}(x_i)$ \Comment{Apply data augmentation to $x_i$}
    \For{$k = 1$ \textbf{to} $K$}
     \State $\bar{u}_{i,k} = \text{Augment}(u_i)$ \Comment{Apply $k$th round of data augmentation to $u_b$}
    \EndFor
    \State $\bar{q}_i = \frac{1}{K} \sum_{k=1}^{K} p_{\text{model}}(y|\bar{u}_{i,k}; \theta)$ \Comment{Compute average predictions across all augmentations of $u_i$}
    \State $q_i = \text{Sharpen}(\bar{q}_i, T)$ \Comment{Apply temperature sharpening to the average prediction}
   \EndFor
   \State $X^\star = ((\bar{x}_i, p_i); i \in (1, \dots, n))$ \Comment{Augmented labeled examples and their labels}
   \State $U^\star = ((\bar{u}_{i,k}, q_i); i \in (1, \dots, n), k \in (1, \dots, K))$ \Comment{Augmented unlabeled examples, guessed labels}
   \State $W = \text{Shuffle}(\text{Concat}(X^\star, U^\star))$ \Comment{Combine and shuffle labeled and unlabeled data}
   \State $X^\prime = (\text{MixUp}(\bar{x}_i, w_i); i \in (1, \dots, |X^\star|))$ \Comment{Apply MixUp to labeled data and entries from $W$}
   \State $U^\prime = (\text{MixUp}(\bar{u}_{i}, w_{i+|X^\star|}); i \in (1, \dots, |U^\star|))$ \Comment{Apply MixUp to unlabeled data and the rest of $W$}
   \State \textbf{return} $X^\prime, U^\prime$
   
 \end{algorithmic}
\end{algorithm}


% VAE, HVAE,LVAE
\section{Variational Autoencoders (VAEs)}
\label{sec:vae}
The \textit{Variational Autoencoder} or VAE for short is generative model that falls into generative modelling (page \pageref{generative-modelling}). 
It is a neural network architecture that is capable of learning a low-dimensional representation of complex high-dimensional data such as 
images, text, or audio. The VAE is a probabilistic model that learns to approximate the true data distribution by using an encoder network to 
map input data into a latent space, and a decoder network to map the latent space back to the original data space. But before we delve into the details of VAE, 
let's explain the term ``Variational Autoencoders'' itself and what it actually represents. The explanation comes in two parts, first we explain 
autoencoders in subsection \ref{subsec:autoencoders} and then variational inference in subsection \ref{subsec:variational-bayes}. The experinced reader can skip those 
introductory parts and go right to subsection \ref{subsec:vaes}.

\subsection{Autoencoders}
\label{subsec:autoencoders}
An autoencoder was first introduced in the 1980s by Hinton \cite{autoencoders-1986}. However, it was not until the advent of deep learning 
and the availability of large amounts of data and computational resources in the 2000s and 2010s that autoencoders became widely used and 
achieved state-of-the-art results in a variety of task \cite{dim-reduction-ae-2006}. An autoencoder is a neural network designed to learn 
identity mapping in unsupervised manner to reconstruct the original input while compressing the information in the ``bottleneck'' layer 
to obtain a compressed representation (see \ref{fig:autoencoder}). Through this, we obtain an effecient dimensionality reduction: The low-dimensional 
latent representation can be used as an embedding vector in various application, such as search or data compression \cite{ae-blog-2018}.

\begin{figure}[t]
    \centering
    \includegraphics[width=\textwidth]{autoencoder.png}
    \caption[Autoencoder]{Ilustration of autonecoder network with two networks: \textit{Encoder} and \textit{Decoder}, each parametrized by learnable parameters.
    Source \cite{ae-blog-2018}}
    \label{fig:autoencoder}
\end{figure}

\subsubsection*{Plato alegory}
The concept of latent variables in generative modeling can be explained using Plato's Allegory of the Cave \cite{plato-alegory}. 
In the allegory, people are confined to a cave and can only see two-dimensional shadows of three-dimensional objects projected onto a wall. 
Similarly, the objects we observe in the world may be generated by higher-level abstract concepts that we can never directly observe. 
These abstract concepts may represent properties such as color, size, and shape. Even though we never see and can not fully comprehend these 
higher-level concepts in all details, we can still reason and draw inferences about them through their manifestation in our lifes. In similar
way, we can approximate the latent representations, which encode the observed data \cite{diffusion-models-blog-2018}.  

However, in generative modeling, we typically seek to learn lower-dimensional latent representations rather than higher-dimensional ones. 
This is because attempting to learn a representation of higher dimension than the observation is often difficult without strong priors. 
Learning lower-dimensional latents can also be viewed as a form of compression, which can uncover semantically meaningful structure describing
observations.

The autoencoder's architecture is composed of two networks:
\begin{itemize}
    \item \textit{Encoder} network, which takes (high-dimensional) input $\boldsymbol{x}$ and maps it into low-dimensional latent code $\boldsymbol{z}$. We denote it as a
    function $\boldsymbol{g}(\cdot)$ parametrized by $\boldsymbol{\phi}$. It's goal is to provide dimensionality reduction, just like 
    any other approaches such as principle component analysis (PCA) or t-SNE. 
    \item \textit{Decoder} network, which takes the code $\boldsymbol{z}$ and recovers the data $\boldsymbol{x}$. We donte it as an function $\boldsymbol{f}(\cdot)$ parametrized by $\boldsymbol{\theta}$.
\end{itemize}
The parameters $\boldsymbol{\theta}$ and $\boldsymbol{\phi}$ are learned simultaneously using stochastic gradient descent (SGD). We encode the input, decode it, and
compute the mean squared error (MSE) loss for each feature in the batch to train the model. By minimizing this loss, we encourage the model
to produce output that is as close as possible to the input.
\begin{align*}
    {\boldsymbol{x}}^\prime &= f_{\boldsymbol{\theta}}(\boldsymbol{g}_{\boldsymbol{\phi}}(\boldsymbol{x})) = f_{\boldsymbol{\theta}}(\boldsymbol{z}) \\
    \mathcal{L}_{\text{MSE}} &= \frac{1}{n} \sum_{i=1}^{n} (\boldsymbol{x}^{\prime}_i - \boldsymbol{x}_i)^2
\end{align*}
Vanilla autoencoders can be prone to overfitting if their capacity is too high relative to the size of the dataset. To improve their
robustness, Vincent et \textit{al.} proposed a \textit{denoising} autoencoder in 2008 \cite{denoising-ae-2008}. This approach involves adding
random noise to the input and then training the model to reconstruct the original signal. By forcing the model to learn the underlying
structure of the data, rather than just memorizing the training examples, denoising autoencoders can achieve better generalization 
performance. The idea of adding noise is today known as dropout technique \cite{dropout-2014}. 

Since then, other architectures have been proposed to improve robustness and prevent overfitting. These include sparse autoencoders
\cite{sparse-ae-2011}, k-sparse autoencoders \cite{ksparse-ae-2014}, and contractive autoencoders \cite{contractive-ae-2011}. The novel 
approach was then defined in 2013 by Kingma and Welling \cite{vae-original-2013} and VAEs were introduced. The key idea was to assume that 
latent space is not deterministic, but stochastic with some distribution $p(\boldsymbol{z})$ over it. The goal of the VAE is to model this 
distribution by variational bayes. 

\subsection{Variational Bayes}

\label{subsec:variational-bayes}
% % 
% \subsubsection*{Variational calculus}
% \todo{remove this section}
% The variational calculus is a broad field of mathematical analysis that shares many similarities with the more familiar continuous optimization
% and differential calculus techniques. In classical physics, it is also known as "Hamilton's principle" and plays a critical role in Lagrangian
% mechanics \cite{lagrangian-mechanics-1998,kulhanek-2016}. The concepts of variational calculus are also extensively used in continuous optimal
% control theory \cite{optimal-control-2004} and many other fields.  In this context, we will introduce the concept of variational calculus 
% indirectly by defining the problem of optimization, highlighting its similarities to differential calculus.

% A real-valued function $f$ defined on a domain $X$ has a global (or absolute) minimum point (minimizer) at $x^\star$, 
% if $f(x^\star) \leq f(x)$ for all $x \in X$. The value of the function at a minimum point is called the \textit{minimum value} of 
% the function, denoted 
% $$
% \min_{x \in X}(f(x)).
% $$ 
% If the domain $X$ is a metric space, then $f$ is said to have a \textit{local minimum point} at $x^\star$, 
% if $f(x^\star) \leq f(x)$ for all $x \in X$ within distance $\epsilon$ of $x^\star$.
% The set of all \textit{minimum points} is denoted as
% $$
% \argmin_{x\in X} = \big\{x\in X \mid f(x) \leq f(x^\prime)\, \forall x^\prime \in X \big\}.
% $$ 
% In both the global and local cases, the concept of a strict extremum can be defined: $x^\star$ is a \textit{strict local maximum point} if 
% there exists some $\epsilon > 0$ such that, for all $x \in X$ within distance $\epsilon$ of $x^\star$ with $x \neq x^\star$, we have 
% $f(x^\star) < f(x)$. Note that a point is a strict global minimum point if and only if it is the unique global minimum point. 

% The goal of the mathematical optimization (also known as mathematical programming) is to find a minimum (or maximum) of a real valued function 
% $f: X^\prime \to \mathbb{R}$ on a subset $X \subset X^\prime$ \cite{werner-opt-2022}. This is very general formulation as the $X$ can be arbitrary.
% In case the $X^\prime = \mathbb{R}^n$, we talk about \textit{continuous optimization} and we use the differential calculus to obtain the solution.
% If the $X^\prime$ is countable, i.e. there exists an injective function from $X^\prime$ to the set of natural numbers:
% $$
% X^\prime  \prec \mathbb{N} = \{1,2,3,...\},
% $$
% we talk about \textit{combinatorical optimization}.
% Lastly, if the $X^\prime$ contains functions itself, e.g. all continous real-valued functions on closed interval $[a,b]$:
% $$
% X^\prime \subset C(a,b),
% $$
% we talk about \textit{calculus of variations}. 

% In variational calculus, we call the small change in the function $x \in X^\prime$ as \textit{variation} and denote it $\delta x$
% (compare it with  \textit{differential} of number $\d x$ from differential calculus). The objective function $f$ is called \textit{functional} as 
% it maps functions to real numbers $\mathbb{R}$. Functionals are often expressed as definite integrals involving functions and their derivatives.
% Even though it is not important for explaining the \textit{variational bayes}, we will mention Euler-Lagrange equations, which plays key role in 
% variational calculus. They are system of second-order ordinary differential equations\cite{intro-variational-calc-2003}.They provide the 
% stationary points (candidates for extrema) of the given functional $f$. Formaly, let 
% \begin{equation*}
%     J(f) = \int_a^b L(x,f(x),f^\prime(x)) \d x 
% \end{equation*}
% be the functional to be minimized. We are looking for a continously differentiable function\cite{smooth-functions-2023} $f \in C^1([a,b])$, which satisfies the boundary conditions $f(a) = A$ and $f(b) = B$.
% We also assume that $L$ is twice continously differentiable. Function $f$ is a stationary point of $J$ if and only if
% \begin{equation*}
%     \frac{\partial L}{\partial f} - \frac{\d}{\d x} \frac{\partial L}{\partial f^\prime} = 0 
% \end{equation*}
% The derivation of Euler-Lagrange equations is straightforward and is shown in many introductory materials, such as \cite{hurak-2021} or \cite{kulhanek-2016}.
% %


The idea of autoencoder was described above, but what does the ``\textit{variational}'' in ``variational autoencoder'' stands for? 
Calculus of variations is a branch of mathematics that deals with finding the optimal solution of a functional, which is a mathematical 
function that maps a set of functions to real numbers. The optimal solution is the function that minimizes or maximizes the functional. 
The calculus of variations has many applications in physics \cite{lagrangian-mechanics-1998}, engineering \cite{optimal-control-2004}, economics, and other fields.
Its development dates back to the 17th century. The Euler-Lagrange equations \cite{intro-variational-calc-2003} play a crucial role in variational calculus by providing the stationary points of a given 
functional that needs to be minimized or maximized.

Now we know what \textit{variations} are, but how do we apply them in the context of bayesian inference? Let's assume we have a probabilistic
graphical model (PGM) or Bayesian network \cite{graphical-models-2023} with some hidden (or unobserved) nodes $H$ and some observed nodes (evidence) $E$. The goal 
of bayesian inference is to compute posterior probability $p(H\mid E)$:
$$
p(H \mid E) = \frac{p(H,E)}{p(E)} = \frac{p(E\mid H) p(H)}{p(E)}
$$ 
where $p(E)$ is marginal density of the evidence:
$$
p(E) = \int_{H} p(H,E)\, \d H
$$
Computing the evidence is for the most of the models intractable due to the integral or higher number of hidden variables (even if they were from
categorical distribuion). The intractability of $p(E)$ is related to the intractability of the
posterior distribution $p(H \mid E)$. Note that the joint distribution $p(H, E)$ is efficient to compute, and that the densities are related by Bayes formula.
Since $p(H, E)$ is tractable to compute, a tractable marginal likelihood $p(E)$ leads to a tractable posterior $p(H \mid E)$, and vice versa \cite{intro-vae-2019,vb-intro-1999}.
 
Variational Bayes (VB) is a technique for approximating complex probability distributions by simpler ones and was introduced by Jordan et al. in 1999 \cite{vb-intro-1999}. 
VB provides a way to approximate the posterior distribution by a simpler one that belongs to a tractable family of distributions, and to do so
by minimizing the Kullback-Leibler (KL) divergence between the true posterior and the approximate one. The KL divergence is a functional with respect 
to approximate tractable distribution $q(H)$ since the true posterior $p(H \mid E)$ is given, hence the connection with \textit{variational calculus}.  
The KL divergence is a measure of the dissimilarity between two probability distributions:
\begin{align*}
q^\star(H) &= \argmin_{q \in \mathcal{Q}} \mathrm{KL}(q(H) || p(H  \mid  E)) \\
\mathrm{KL}(q(H) || p(H \mid E)) &= \int q(H) \log \big[ \frac{q(H)}{p(H \mid E)} \big]\,\d H,
\end{align*}
where $\mathcal{Q}$ is the family of tractable distributions, and $\mathrm{KL}$ is the KL divergence. It should be noted that there are other 
non-optimization based methods to do such approximate inference, such as MCMC \cite{wiki-mcmc-2023}. 

As the KL divergence is not symmetrical, one could ask why we have defined the optimization task as the reverse KL-divergence and not the other way
around, i.e. forward KL divergence $\mathrm{KL}(p(H \mid E) \mid q(H))$. Both cases are illustrated in figure \ref{fig:forward-reverse}. The reverse 
KL~divergence minimization results in $q$ \textit{under-estimating} $p$, which can be perceived as a safe choice. This choice ensures that sampling
from found $q$ provides values which are plausible under original $p$. For a thorough explanation, we refer to the literature 
\cite{another-vb-intro-2021}.
\begin{figure}[t]
    \centering
    \includegraphics[width=\textwidth]{fwd-rvrs-kl.png}
    \caption[Forward vs reverse KL divergence]{Ilustration of forward vs reverse KL-divergence on a bimodal distribution. The blue and the red contours represent the target $p$ 
    and the unimodal approximation $q$, respectively.In (a), the forward KL-divergence minimization is visulized with $q$ trying to cover up $p$.
    The (b) and (c) shows the reverse KL-divergence where $q$ locks on to one of the two modes. Source \cite{another-vb-intro-2021}}
    \label{fig:forward-reverse}
\end{figure}
We will now relabel our variables to follow
the standard notation used in deep learning literatue, where hidden variables $H$ are known as latent $\mathbf{z}$ and observed $E$ as features $\mathbf{x}$. 
In this setting, we want to optimize:
\begin{align}
    \label{eq:variational_bayes}
    q^\star(\mathbf{z}) &= \argmin_{q \in \mathcal{Q}} \mathrm{KL}(q(\mathbf{z}) || p(\mathbf{z} \mid \mathbf{x})) \\
    \mathrm{KL}(q(\mathbf{z}) || p(\mathbf{z} \mid \mathbf{x})) &= \int_\mathbf{z} q(\mathbf{z}) \log \big[ \frac{q(\mathbf{z})}{p(\mathbf{z}\mid\mathbf{x})} \big]\,\d \mathbf{z} \\
    &= \int_\mathbf{z} \big[q(\mathbf{z}) \log q(\mathbf{z})\big] \d \mathbf{z} - \int_\mathbf{z}  \big[q(\mathbf{z}) \log p(\mathbf{z}\mid \mathbf{x})\big] \d \mathbf{z} \\
    &= \EX_q \big[\log q(\mathbf{z})\big] - \EX_q \big[\log p(\mathbf{z}\mid \mathbf{x})\big] \\
    &= \EX_q \big[\log q(\mathbf{z})\big] - \EX_q \Big[\log \big( \frac{p(\mathbf{x},\mathbf{z})}{\mathbf{x}}\big) \Big] \\
    &= \EX_q \big[\log q(\mathbf{z})\big] - \EX_q \big[\log p(\mathbf{x},\mathbf{z})\big] + \EX_q \big[\log p(\mathbf{x})\big] \\
    &= \EX_q \big[\log q(\mathbf{z})\big] - \EX_q \big[\log p(\mathbf{x},\mathbf{z})\big] + \log p(\mathbf{x})
\end{align}
where the $\EX_q \big[\log p(\mathbf{x})\big] = \log p(\mathbf{x})$ because the $p(\mathbf{x})$ does not depend on $q(\mathbf{x})$. 
We can not directly optimize the KL divergence since the evidence $p(\mathbf{x})$ is intractable, however it is constant (for given datset).
If we rearange the last equation, we obtain 
\begin{align}
\label{eq:elbo}
\log p(\mathbf{x}) - \mathrm{KL}(q(\mathbf{z}) || p(\mathbf{z} \mid  \mathbf{x})) &= \EX_q \big[\log p(\mathbf{x},\mathbf{z})\big] - \EX_q \big[\log q(\mathbf{z})\big] \\
    &= \mathrm{ELBO}(q) 
\end{align}
where the left hand side (lhs) of the equation is called evidence lower bound (ELBO), since it is truly a lower bound on the logarithm of evidence 
$p(\mathbf{x})$. This is clear to see as the KL divergnce is always positive. As the $\log p(\mathbf{x})$ is constant, maximizing the rhs is 
equal to minimzing the KL-divergence:
\begin{align*} 
    q^\star(\mathbf{z}) &= \argmin_{q \in \mathcal{Q}} \mathrm{KL}(q(\mathbf{z}) || p(\mathbf{z} \mid  \mathbf{x})) \\
                        &= \argmax_{q \in \mathcal{Q}} \mathrm{ELBO}(q) \\
                        &= \argmax_{q \in \mathcal{Q}} \bigl[ \EX_q \big[\log p(\mathbf{x},\mathbf{z})\big] - \EX_q \big[\log q(\mathbf{z})\big] \bigr]  \\
\end{align*}
A commonly made assumption regarding the variational distribution $q(\mathbf{z})$ is that it factorizes over a partition of the latent variables, 
such that the latent variables can be partitioned into subsets $\mathbf{z}_1,\dots,\mathbf{z}_m$. Specifically, we can write:
$$
q(\mathbf{z}) = \prod_{i=1}^m q_i(\mathbf{z}_i\mid \mathbf{x})
$$
This assumption is known as a \textit{mean field approximation}. It can be shown (see \cite{intro-variational-blog-2019}) that the following 
equation holds for optimal $q^\star_j$: 
$$
\log q^\star_j(\mathbf{z}_j\mid \mathbf{x}) = \EX_{i \neq j} \log p(\mathbf{x},\mathbf{z}) + \text{constant}
$$
where $\EX_{{i\neq j}}[\log p({\mathbf {z}},{\mathbf {x}})]$ represents the expected value of the logarithm of the joint probability 
of the observed data and latent variables, taken over all variables that are not included in the current partition.The constant is 
only a normalizing constant, so $q^\star_j$ is truly a probability distribution.This leads to iterative 
algorithm of block-cordinate ascent on ELBO with initial random guess (see algorithm \ref{alg:cavi}). Furthermore, Neal and Hinton
have shown in \cite{em-as-vb-1998} that the Expectation Maximization (EM) alogrithm (proposed by Dempster et \textit{al.} \cite{em-alg-1977}) 
can be seen as such ascent on ELBO. 

\begin{algorithm}
    \caption[CAVI algorithm]{Coordinate Ascent Variational Inference (CAVI) Source: \cite{intro-variational-blog-2019}}
    \label{alg:cavi}
    \begin{algorithmic}[2]
      \State \textbf{Input:} A model $p(\mathbf{x},\mathbf{z})$, a dataset $\mathbf{x}$
      \State \textbf{Initialize:} Variational factors $q_j(\mathbf{z}_j)$
      \While{the $\mathrm{ELBO}$ has not converged}
        \For{$j \in \{1,\dots,m\}$}
          \State Set $q_j(z_j) \gets \frac{\exp{\EX_{i \neq j} [\log p(\mathbf{z},\mathbf{x})]}}{\int_{\mathbf{z}_j}\exp{\EX_{i \neq j} [\log p(\mathbf{z},\mathbf{x})]}\d \mathbf{z}_j}$
        \EndFor
        \State Compute $\mathrm{ELBO}(q) \gets \EX \bigl[\log(p(\mathbf{x},\mathbf{z}))\bigr] - \EX \bigl[ \log q(\mathbf{z}) \bigr]$
      \EndWhile
      \State \textbf{return} $\prod_{j=1}^m q_j(\mathbf{z}_j)$
    \end{algorithmic}
\end{algorithm}

\subsection{VAE}
\label{subsec:vaes}
The variational autoencoders were first introduced by Kingma and Welling in 2013 \cite{vae-original-2013}. Since its introduction, the VAE 
has been widely used in research and industry, and has inspired many other generative models. It continues to be an active area of research, with 
ongoing efforts to improve its performance and applicability to different domains. This subsection was mainly inspired by \cite{intro-vae-2019}.

\subsubsection{Deep Latent Variable Models}
A deep latent variable model (DLVM) is a type of probabilistic graphical model or Bayesian network where some variables are hidden or latent \cite{intro-vae-2019}.
These models use neural networks to parameterize the distributions of the its variables, enabling very complex marginal distributions (evidence)
$p_{\boldsymbol{\theta}}(\mathbf{x})$ even if each factor in the directed model (prior $p_{\boldsymbol{\theta}}(\mathbf{z})$ or conditional $p_{\boldsymbol{\theta}}(\mathbf{x}\mid \mathbf{z})$)
is relatively simple. This expresivity maskes them attractive for approximating complicated true distributions $p^\star(\mathbf{x})$. 
The simplest DLVM is one that is factored as with following structure:
$$
p_{\boldsymbol{\theta}}(\mathbf{x},\mathbf{z}) = p_{\boldsymbol{\theta}}(\mathbf{z})p_{\boldsymbol{\theta}}(\mathbf{x}\mid \mathbf{z})
$$
where $p_{\boldsymbol{\theta}}(\mathbf{z})$ and/or $p_{\boldsymbol{\theta}}(\mathbf{x}\mid \mathbf{z})$ are specified, i.e. their family of distribution is fixed and its distribution
parameters $\boldsymbol{\eta}$ are parametrized by neural network:
\begin{align*}
\boldsymbol{\eta} &= \mathrm{NeuralNet}_{\boldsymbol{\theta}}(\text{Pa}(\mathbf{x})) \\
p_{\boldsymbol{\theta}}(\mathbf{x}\mid \text{Pa}(\mathbf{x})) &= p_{\boldsymbol{\theta}}(\boldsymbol{x}\mid \boldsymbol{\eta})
\end{align*}
or even both their family and distribuion parameters are fixed, e.g. 
$$
p_{\boldsymbol{\theta}}(\mathbf{z}) = p(\mathbf{z}) = \mathcal{N}(\mathbf{z},0,\mathbf{I})
$$
The $\text{Pa}(\mathbf{x}))$ corresponds to all parents of node $\mathbf{x}$ in Bayesian Network.
Such DLVM is visualized in picture \ref{fig:vae_pgm} as a Bayesian network. We will later see that 
it is exactly this model that is under consideration when talking about VAE. 

\begin{example}[DLVM for multivariate Bernoulli data]
    A simple example of DLVM for binary model $\mathbf{x} \in \{0,1\}^{D}$ used in \cite{vae-ssl-dgm-2014} has following structure:
    Latent space $\mathbf{z}$ is fixed as a spherical Gaussian distribution and conditional probability is modeled as
    a factorized Bernoulli:
    \begin{align*}
        p(\mathbf{z}) &= \mathcal{N}(\mathbf{z},0,\mathbf{I})\\
        \mathbf{p} &= \mathrm{DecoderNet}_{\boldsymbol{\theta}}(\mathbf{z}) \\
        \log p(\mathbf{x}\mid \mathbf{z}) &= \sum_{i=1}^D log(x_j \mid  \mathbf{z}) = \sum_{i=1}^D \log \mathrm{Bern}(x_i,p_i)\\
        &= \sum_{i=1}^D x_i \log p_j + (1-x_j) \log(1-p_j)\\
    \end{align*}
    where $p_j \in [0,1]\,\forall p_j \in \mathbf{p}$ and $\mathrm{Bern}(\cdot,p)$ is the probability mass function (PMF)
    of the Bernoulli distribution.
\end{example}

\begin{figure}[tb]
    \centering
    \includegraphics[width=\textwidth]{vae_pgm.png}
    \caption[VAE as Bayesian network]{Ilustration of VAE as Bayesian Network. Solid lines denote generative capabilities 
    $p_{\boldsymbol{\theta}}(\mathbf{x}\mid \mathbf{z})p_{\boldsymbol{\theta}}(\mathbf{z})$, dashed lines denote the variational approximation $q_{\boldsymbol{\phi}}(\mathbf{z}\mid \mathbf{x})$
    to the intractable posterior $p_{\boldsymbol{\theta}}(\mathbf{z}\mid \mathbf{x})$. The variational parameters ${\boldsymbol{\phi}}$ are learned jointly with the generative model pa-
    rameters ${\boldsymbol{\theta}}$. Source \cite{vae-original-2013}}
    \label{fig:vae_pgm}
\end{figure}

Up until now, we have only discussed DLVMs, which involve using neural networks in Bayesian networks. However, DLVMs face the same issues as 
other Bayesian models discussed in the Variational Bayes section, such as the inability to compute the evidence $p(\mathbf{x})$ due to 
intractabilities. To overcome this, approximate inference techniques such as Expectation Maximization (EM) (see \ref{alg:cavi}) or 
Monte Carlo Markov Model (MCMC)  can be used to approximate the posterior $p_{\boldsymbol{\theta}}(\mathbf{z}\mid \mathbf{x})$ and the marginal 
likelihood $p_{\boldsymbol{\theta}}(\mathbf{x})$ \cite[appx.~A2]{intro-vae-2019}.  However, these traditional inference methods are computationally expensive,
often requiring per-datapoint optimization loops and are not well-suited for datasets with a large number of examples $N$. Moreover, they tend
to yield poor posterior approximations. Thus, there is a need for more efficient procedures.

\subsubsection{Approximate Posterior}
\label{subsubsec:vae_encoder}
The VAEs frameworks introduces two models, encoder and decoder (see figure \ref{fig:vae_pgm} ) and also propose an computationally effecient algorithm for optimizing the parameters jointly
using SGD.

To address the intractabilities of DLVM, VAEs utilize the \textit{recognition model} $q_{\boldsymbol{\phi}}(\mathbf{z}\mid \mathbf{x})$, also known as an 
\textit{inference model} or an \textit{encoder}, to approximate the intractable $p_{\boldsymbol{\theta}}(\mathbf{z}\mid \mathbf{x})$. 
While the encoder, similar 
to DLVM, can be any Bayesian network, in vanilla VAE, it is typically a simple network with $\mathbf{x}$ as input, which produces the parameter for
the distribution for the latent $\mathbf{z}$. We denote the parameters of this network with ${\boldsymbol{\phi}}$ to differentiate them from ${\boldsymbol{\theta}}$, and refer
to them as \textit{variational}, as the goal of the encoder is to approximate the conditional distribution $p_{\boldsymbol{\theta}}(\mathbf{z}\mid \mathbf{x})$:
$$
q_{\boldsymbol{\phi}}(\mathbf{z}\mid \mathbf{x}) \approx p_{\boldsymbol{\theta}}(\mathbf{z}\mid \mathbf{x})
$$
It is important to note that, unlike the approximate posterior in mean-field variational inference, the encoder is not necessarily factorial, and
its parameters ${\boldsymbol{\phi}}$ are not computed from a closed-form expectation \cite{vae-original-2013}. Instead, VAEs' framework introduces a method for 
jointly learning the recognition model parameters ${\boldsymbol{\phi}}$ and the generative model parameters ${\boldsymbol{\theta}}$.  This approach is known as \textit{amortized 
variational inference} \cite{amortized-inference-2014}, which avoids a per-datapoint optimization loop and leverages the efficiency of SGD \cite{intro-vae-2019}.
\subsubsection{ELBO objective}
The objective to minimize is an ELBO (defined in \ref{eq:elbo}), like in any other bayes variational methods. For any selection of recognition model $q_{\boldsymbol{\phi}}(\mathbf{z}\mid \mathbf{x})$,
we can write:
\begin{align}
    \label{eq:vae}
    \log p_{\boldsymbol{\theta}}(\mathbf{x})  &= \EX_{q_{\boldsymbol{\phi}}(\mathbf{z}\mid \mathbf{x})} \big[\log p_{\boldsymbol{\theta}}(\mathbf{x},\mathbf{z})\big] - \EX_{q_{\boldsymbol{\phi}}(\mathbf{z}\mid \mathbf{x})} \big[\log q_{\boldsymbol{\phi}}(\mathbf{z}\mid \mathbf{x})\big] &+ \mathrm{KL}(q_{\boldsymbol{\phi}}(\mathbf{z}\mid \mathbf{x}) || p_{\boldsymbol{\theta}}(\mathbf{z} \mid  \mathbf{x})) \\
    \log p_{\boldsymbol{\theta}}(\mathbf{x})  &= \mathrm{ELBO}(q_{\boldsymbol{\phi}}(\mathbf{z}\mid \mathbf{x})) &+ \mathrm{KL}(q_{\boldsymbol{\phi}}(\mathbf{z}\mid \mathbf{x}) || p_{\boldsymbol{\theta}}(\mathbf{z} \mid  \mathbf{x}))
\end{align}
This equation provides important insight into the divergence $\mathrm{KL}(q_{\boldsymbol{\phi}}(\mathbf{z}\mid \mathbf{x}) || p_{\boldsymbol{\theta}}(\mathbf{z} \mid  \mathbf{x}))$. 
Firstly, it measures how well the approximating distribution $q_{\boldsymbol{\phi}}(\mathbf{z}\mid \mathbf{x})$ fits the intractable distribution of decoder 
model $p_{\boldsymbol{\theta}}(\mathbf{z} \mid  \mathbf{x})$. Secondly, it quantifies the gap between the ELBO and the log likelihood of the marginal likelihood 
$\log p_{\boldsymbol{\theta}}(\mathbf{x})$, also knwon as \textit{tightness} of the bound. A tighter bound indicates a better fit between the approximate 
and model posterior distributions. Therefore maximizing the $\mathrm{ELBO}$ w.r.t to ${\boldsymbol{\theta}}$ and ${\boldsymbol{\phi}}$ can be seen as a concurent task of 
improving the log evidence $p_{\boldsymbol{\theta}}(x)$ under the model and also improving the fit of variational $q_{\boldsymbol{\phi}}(\mathbf{z}\mid \mathbf{x})$ to 
$p_{\boldsymbol{\theta}}(\mathbf{z} \mid  \mathbf{x}))$:
\begin{align}
    \label{eq:opt_vae}
    \argmax_{{\boldsymbol{\theta}},{\boldsymbol{\phi}}} \mathrm{ELBO}(q_{\boldsymbol{\phi}}(\mathbf{z}\mid \mathbf{x})) &= \argmax_{{\boldsymbol{\theta}},{\boldsymbol{\phi}}} \bigr[ \log p_{\boldsymbol{\theta}}(\mathbf{x}) - \mathrm{KL}(q_{\boldsymbol{\phi}}(\mathbf{z}\mid \mathbf{x}) || p_{\boldsymbol{\theta}}(\mathbf{z} \mid  \mathbf{x})) \bigl] 
\end{align}
In order to use SGD, we need to compute the derivatives of the $\mathrm{ELBO}$ objective with respect to the model parameters ${\boldsymbol{\theta}}$ and ${\boldsymbol{\phi}}$. 
The $\mathrm{ELBO}$ objective of a (mini)-batch is defined as the sum of terms for each data point in the (mini)-batch $\mathcal{B}$:
$$
\mathrm{ELBO}(\mathcal{B}) = \sum_{\mathbf{x} \in \mathcal{B}} \mathrm{ELBO}(q_{\boldsymbol{\phi}}(\mathbf{z}\mid \mathbf{x}))
$$
where the data point term is defined as
$$
\mathrm{ELBO}(q_{\boldsymbol{\phi}}(\mathbf{z}\mid \mathbf{x})) = \EX_{q_{\boldsymbol{\phi}}(\mathbf{z}\mid \mathbf{x})} \big[\log p_{\boldsymbol{\theta}}(\mathbf{x},\mathbf{z})\big] - \EX_{q_{\boldsymbol{\phi}}(\mathbf{z}\mid \mathbf{x})} \big[\log q_{\boldsymbol{\phi}}(\mathbf{z}\mid \mathbf{x})\big].
$$

This term depends on the data point $\mathbf{x}$ and the network parameters ${\boldsymbol{\theta}}$ and ${\boldsymbol{\phi}}$ that correspond to the distributions $p_{\boldsymbol{\theta}}$ and $q_{\boldsymbol{\phi}}$, respectively.
The ELBO is sometimes rearanged into:
\begin{align}
    \mathrm{ELBO}(q_{\boldsymbol{\phi}}(\mathbf{z}\mid \mathbf{x})) &= \EX_{q_{\boldsymbol{\phi}}(\mathbf{z}\mid \mathbf{x})} \big[\log p_{\boldsymbol{\theta}}(\mathbf{x}\mid \mathbf{z}) + \log p_{\boldsymbol{\theta}}(\mathbf{z}) - \log q_{\boldsymbol{\phi}}(\mathbf{z}\mid\mathbf{x})\big] \notag\\
    &= \EX_{q_{\boldsymbol{\phi}}(\mathbf{z}\mid\mathbf{x})} \big[\log p_{\boldsymbol{\theta}}(\mathbf{x}\mid\mathbf{z}) \big] - \EX_{q_{\boldsymbol{\phi}}(\mathbf{z}\mid\mathbf{x})} \big[  \log \frac{q_{\boldsymbol{\phi}}(\mathbf{z}\mid\mathbf{x})}{p_{\boldsymbol{\theta}}(\mathbf{z})} \big] \notag\\
    &= \underbrace{\EX_{q_{\boldsymbol{\phi}}(\mathbf{z}\mid\mathbf{x})} \big[\log p_{\boldsymbol{\theta}}(\mathbf{x}\mid\mathbf{z}) \big]}_{\text{recognition term}} - \underbrace{\mathrm{KL}(q_{\boldsymbol{\phi}}(\mathbf{z}\mid\mathbf{x}) ||p_{\boldsymbol{\theta}}(\mathbf{z}))}_{\text{variational regularization term}} \label{eq:elbo_terms}
\end{align}
Although the individual data point $\mathrm{ELBO}$ and its gradient $\nabla_{{\boldsymbol{\theta}},{\boldsymbol{\phi}}}\,\mathrm{ELBO}(\mathrm{x},{\boldsymbol{\theta}},{\boldsymbol{\phi}})$ are generally not tractable, VAEs provide
means and assumptions under which we can perform minibatch SGD. Namely we will show that good unbiased estimators $\hat{\nabla}_{{\boldsymbol{\theta}},{\boldsymbol{\phi}}}\,\mathrm{ELBO}(\mathrm{x},{\boldsymbol{\theta}},{\boldsymbol{\phi}})$ 
exist under the assumption that we know how to reparametrize the $q_{\boldsymbol{\phi}}$ distribuion to remove the ${\boldsymbol{\phi}}$ influence on sampling. This is also known as \textit{reparametrization trick}
and is a key achievment of VAEs frameworks. 

Say we want to compute gradient w.r.t. to decoder parameters ${\boldsymbol{\theta}}$, i.e. 
\begin{align}
    \nabla_{{\boldsymbol{\theta}}}\,\mathrm{ELBO}(\mathbf{x},{\boldsymbol{\theta}},{\boldsymbol{\phi}}) &= \nabla_{{\boldsymbol{\theta}}} \EX_{q_{\boldsymbol{\phi}}(\mathbf{z}\mid\mathbf{x})} \big[\log p_{\boldsymbol{\theta}}(\mathbf{x},\mathbf{z}) - \log q_{\boldsymbol{\phi}}(\mathbf{z}\mid\mathbf{x})\big] \notag \\
        &= \EX_{q_{\boldsymbol{\phi}}(\mathbf{z}\mid\mathbf{x})} \nabla_{{\boldsymbol{\theta}}} \big[\log p_{\boldsymbol{\theta}}(\mathbf{x},\mathbf{z}) - \log q_{\boldsymbol{\phi}}(\mathbf{z}\mid\mathbf{x})\big] \label{eq:sec_line} \\
        &\simeq  \nabla_{{\boldsymbol{\theta}}} \big[\log p_{\boldsymbol{\theta}}(\mathbf{x},\tilde{\mathbf{z}}) - \log q_{\boldsymbol{\phi}}(\boldsymbol{\tilde{z}}\mid\mathbf{x})\big] \notag\\
        &\simeq \nabla_{{\boldsymbol{\theta}}} \big[\log p_{\boldsymbol{\theta}}(\mathbf{x},\tilde{\mathbf{z}})\big] \label{eq:monte_carlo}
\end{align}
The $\simeq$ symbol means that one of the two sides is an unbiased estimator of the other side. The rhs is a random variable with some source of the noise and two sides are equal
when averaged over the noise distribuion. In this case, the noise distribution is $q_{\boldsymbol{\phi}}(\mathbf{z}\mid \mathbf{x})$ and $\tilde{\mathbf{z}}$ is sampled from it. In other words, the last 
line (eq. \ref{eq:monte_carlo}) is a Monte Carlo estimator of the second line (eq. \ref{eq:sec_line}) \cite{intro-vae-2019}.

The unbiased gradient w.r.t. to the encoder parameters ${\boldsymbol{\theta}}$ is slightly difficult to obtain as the expectation of the $\mathrm{ELBO}$ is taken w.r.t. the variational distribuion 
$q_{\boldsymbol{\phi}}(\mathbf{z}\mid \mathbf{x})$, which is a function of ${\boldsymbol{\phi}}$, i.e.
\begin{align}
    \nabla_{{\boldsymbol{\phi}}}\,\mathrm{ELBO}(\mathbf{x},{\boldsymbol{\theta}},{\boldsymbol{\phi}}) &= \nabla_{{\boldsymbol{\phi}}} \EX_{q_{\boldsymbol{\phi}}(\mathbf{z}\mid \mathbf{x})} \big[\log p_{\boldsymbol{\theta}}(\mathbf{x},\mathbf{z}) - \log q_{\boldsymbol{\phi}}(\mathbf{z}\mid \mathbf{x})\big] \\
                                                         &\neq  \EX_{q_{\boldsymbol{\phi}}(\mathbf{z}\mid \mathbf{x})} \nabla_{{\boldsymbol{\phi}}} \big[\log p_{\boldsymbol{\theta}}(\mathbf{x},\mathbf{z}) - \log q_{\boldsymbol{\phi}}(\mathbf{z}\mid \mathbf{x})\big]
\end{align}
Nevertheless, in case of continuous latent variables, we can use \textit{reparametrization trick} \cite{renedze-backprop-2014} as we will show now \cite{intro-vae-2019}.

\subsubsection{Reparametrization trick}
By parametrizing the random variable $\mathbf{z} \sim q_{\boldsymbol{\phi}}(\mathbf{z}\mid \mathbf{x})$ as a differentiable and invertible mapping of another random variable $\boldsymbol{\epsilon}$ that is independent
of both the data $\mathbf{x}$ and the variational parameters ${\boldsymbol{\phi}}$, we can obtain an unbiased gradient and differentiate the evidence lower bound (ELBO) objective with respect to ${\boldsymbol{\phi}}$. 
Specifically, we have 
\begin{equation*}
    \mathbf{z} = \mathbf{g}(\boldsymbol{\epsilon},{\boldsymbol{\phi}},\mathbf{x}),
\end{equation*}
where $\boldsymbol{\epsilon}$ is randomly sampled from some distribution $p(\boldsymbol{\epsilon})$. Using this mapping, we can separate the source of noise from the variational parameters and 
obtain the Monte Carlo estimator of the gradient. i.e.
\begin{align*}
    \nabla_{{\boldsymbol{\phi}}} \EX_{q_{\boldsymbol{\phi}}(\mathbf{z}\mid \mathbf{x})} \big[\log \frac{p_{\boldsymbol{\theta}}(\mathbf{x},\mathbf{z})}{q_{\boldsymbol{\phi}}(\mathbf{z}\mid \mathbf{x})}\big] 
    &= \nabla_{{\boldsymbol{\phi}}} \EX_{p(\boldsymbol{\epsilon})} \big[\log p_{\boldsymbol{\theta}}(\mathbf{x},\mathbf{g}(\boldsymbol{\epsilon},{\boldsymbol{\phi}},\mathbf{x})) - \log q_{\boldsymbol{\phi}}(\mathbf{g}(\boldsymbol{\epsilon},{\boldsymbol{\phi}},\mathbf{x})\mid \mathbf{x})\big]\\
    &=  \EX_{p(\boldsymbol{\epsilon})} \nabla_{{\boldsymbol{\phi}}} \big[\log p_{\boldsymbol{\theta}}(\mathbf{x},\mathbf{g}(\boldsymbol{\epsilon},{\boldsymbol{\phi}},\mathbf{x})) - \log q_{\boldsymbol{\phi}}(\mathbf{g}(\boldsymbol{\epsilon},{\boldsymbol{\phi}},\mathbf{x})\mid \mathbf{x})\big] \\
    &\simeq \nabla_{{\boldsymbol{\phi}}} \big[\log p_{\boldsymbol{\theta}}(\mathbf{x},\tilde{\mathbf{z}}) - \log q_{\boldsymbol{\phi}}(\tilde{\mathbf{z}}\mid \mathbf{x})\big],
\end{align*}
where $\tilde{\mathbf{z}} = \mathbf{g}(\tilde{\boldsymbol{\epsilon}},{\boldsymbol{\phi}},\mathbf{x})$  for randomly sampled $\tilde{\boldsymbol{\epsilon}} \sim p(\boldsymbol{\epsilon})$. 
With this procedure, we can now differentiate the ELBO objective with respect to ${\boldsymbol{\phi}}$, which was not possible before, the situation is visualized in figure \ref{fig:reparam_trick}. The limitations of this procedures are, that we have to know the mapping $\mathbf{g}$.
\begin{figure}[t]
    \centering
    \includegraphics[width=\textwidth]{reparametrization_trick.png}
    \caption[Reparametrization trick]{Ilustration of reparametrization trick. The encoder parameters ${\boldsymbol{\phi}}$ affect the objective $f$ indirectly through the random variable $\mathbf{z} \sim  q_{\boldsymbol{\phi}}(\mathbf{z}\mid \mathbf{x})$.
    We can not compte the gradient $\nabla_{\boldsymbol{\phi}} f$ in the original setting (left), as we can not propagate the dependency of ${\boldsymbol{\phi}}$ through the sampling process. Thanks to parameterization trick, we 
    can first sample the $\epsilon$ and then compute $\mathbf{g}(\boldsymbol{\epsilon},{\boldsymbol{\phi}},\mathbf{x})$ which enables us to compute the gradients $\nabla_{\boldsymbol{\phi}} f$.
    Source \cite{intro-vae-2019}}
    \label{fig:reparam_trick}
\end{figure}
For the Gaussian distribuion the particular mapping is the following:
$$
\mathbf{z} = \boldsymbol{\mu} + \boldsymbol{\epsilon} \odot \boldsymbol{\sigma}
$$
where $\boldsymbol{\mu}$ and $\boldsymbol{\sigma}$ are the mean and standard deviation of the Gaussian distribution, respectively, 
$\boldsymbol{\epsilon}$ is a vector of independent samples from a standard normal distribution, and $\odot$ denotes element-wise 
multiplication. The final algorithm for computing SGD for the ELBO loss is provided in \ref{alg:sgd}.
\begin{algorithm}[H]
    \caption[SGD for VAE]{Stochastic gradient descent for VAE. (a.k.a. Auto-Encoding Variational Bayes (AEVB) Algorithm )}
    \label{alg:sgd}
    \begin{algorithmic}[1]
    \State \textbf{Input}: Dataset $\mathcal{D}$, Inference model $q_{\boldsymbol{\phi}}(\mathbf{z}\mid \mathbf{x})$, Generative model $p_{\boldsymbol{\theta}}(\mathbf{x},\mathbf{z})$
    \State \textbf{Return}: Learned parameters $\boldsymbol{\theta}, \boldsymbol{\phi}$
    \State $(\boldsymbol{\theta}, \boldsymbol{\phi}) \leftarrow$ Initialize parameters
    \While{SGD not converged}
    \State $M \sim \mathcal{D}$ \Comment{Random minibatch of data}
    \State $\tilde{L}_{\theta,\phi} = 0$
    \For{$\boldsymbol{m}$ \textbf{in} $M$}
        \State Sample $\boldsymbol{\epsilon} \sim p(\boldsymbol{\epsilon})$ 
        \State Apply encoder $\boldsymbol{\eta} = \mathrm{NeuralNet}_{\boldsymbol{\phi}}(\boldsymbol{m})$
        \State Reparametrize to obtain sample $\tilde{\boldsymbol{z}}$ = $\boldsymbol{g}(\boldsymbol{\eta},\boldsymbol{m},\boldsymbol{\epsilon})$ \Comment{Via reparametrization trick}
        \State Apply decoder on $\tilde{\boldsymbol{z}}$ and compute $\log(p_{\boldsymbol{\theta}}(\mathbf{x}=\boldsymbol{m}\mid \mathbf{z}=\tilde{\boldsymbol{z}})) + p_{\boldsymbol{\theta}}(\boldsymbol{z}=\tilde{\boldsymbol{z}})$
        \State Compute negative log: $-\log q_{\boldsymbol{\phi}}(\boldsymbol{z}=\tilde{\boldsymbol{z}}\mid \boldsymbol{x}=\boldsymbol{m})$ \Comment{See subsection \ref{subsubsec:log_q}}
        \State $\tilde{L}_{\theta,\phi} \mathrel{+}= \log(p_{\boldsymbol{\theta}}(\mathbf{x}=\boldsymbol{m},\mathbf{z}=\tilde{\boldsymbol{z}})) - \log q_{\boldsymbol{\phi}}(\boldsymbol{z}=\tilde{\boldsymbol{z}}\mid \boldsymbol{x}=\boldsymbol{m})$ \Comment{Accumulate loss}
    \EndFor
    \State Compute gradients $\nabla_{\theta,\phi} \,\tilde{L}_{\theta,\phi}(M, \boldsymbol{\epsilon})$
    \State Update $\theta$ and $\phi$ using SGD optimizer
    \EndWhile
    \end{algorithmic}
\end{algorithm}

\subsubsection{Evaluation of $\log_{\boldsymbol{\phi}}(\boldsymbol{z}\mid \boldsymbol{x})$}
\label{subsubsec:log_q}
The computation of the estimator of the ELBO requires computation of the density log: (see algorithm \ref{alg:sgd})
$$
\log_{\boldsymbol{\phi}}(\boldsymbol{z}=\tilde{\boldsymbol{z}}\mid \boldsymbol{x}=\boldsymbol{m}).
$$ 
The $\tilde{\boldsymbol{z}}$ is reparametrized through the \textit{reparametrization trick}
$$
\tilde{\boldsymbol{z}} = \mathbf{g}(\tilde{\boldsymbol{\epsilon}},{\boldsymbol{\phi}},\mathbf{x}),\quad \tilde{\boldsymbol{\epsilon}} \sim p(\boldsymbol{\epsilon})
$$
and as long as the $\mathbf{g}(\cdot)$ is invertible, we can express the density of $\boldsymbol{z}$ through the density of $\boldsymbol{\epsilon}$:
$$
q_{\boldsymbol{\phi}}(\mathbf{z}\mid \mathbf{x}) = \log p(\boldsymbol{\epsilon}) - \log d_{\boldsymbol{\phi}}(\boldsymbol{x},\boldsymbol{\epsilon})
$$
where $d_{\boldsymbol{\phi}}(\boldsymbol{x},\boldsymbol{\epsilon})$ is the log of the absolute value of the determinant of the Jacobian 
matrix \cite{intro-vae-2019}:
\begin{align*}
d_{\boldsymbol{\phi}}(\boldsymbol{x},\boldsymbol{\epsilon}) &= \log |\det J(\boldsymbol{z}, \boldsymbol{\epsilon})|\\
J(\boldsymbol{z}, \boldsymbol{\epsilon}) = \frac{\partial \boldsymbol{z}}{\partial \boldsymbol{\epsilon}} &=
\begin{bmatrix}
\frac{\partial z_1}{\partial \epsilon_1} & \frac{\partial z_1}{\partial \epsilon_2} & \cdots & \frac{\partial z_1}{\partial \epsilon_k} \\
\frac{\partial z_2}{\partial \epsilon_1} & \frac{\partial z_2}{\partial \epsilon_2} & \cdots & \frac{\partial z_2}{\partial \epsilon_k} \\
\vdots & \vdots & \ddots & \vdots \\
\frac{\partial z_k}{\partial \epsilon_1} & \frac{\partial z_k}{\partial \epsilon_2} & \cdots & \frac{\partial z_k}{\partial \epsilon_k}
\end{bmatrix}
\end{align*}
One can create a very flexible transformations $\boldsymbol{g}(\cdot)$ which results in more flexible inference model 
$\log_{\boldsymbol{\phi}}(\boldsymbol{z}\mid\boldsymbol{x})$ improving the capacity of the model. Between those transformations
belong factorized Gaussian posterior \cite{intro-vae-2019}, Normalizing flows (NF) \cite{nf-2015} or Inverse autoregresive transformations and 
Inverse autoregresive flows (IAF) \cite{iaf-2016}. Other possibility is to extend the latent space with auxiliary latent variables
\cite{aux-var-2016} which leads to hiearchical variational autoencoders (HVAE) firstly introduced in \cite{lvae-2016}.

\subsection{Issues with VAEs}
As every method, the vanilla VAEs have also its weak points, which were identified and posses an open question with active research:
\begin{itemize}
    \item \textbf{Bluriness of the images}: If the VAE model does not have the enough capacity to model the underlying true distribution properly,
    the variance of decoder $p_{\boldsymbol{\theta}}(\boldsymbol{x}\mid \boldsymbol{z})$ will end up larger then the variance of the encoder 
    $$
    q_{\mathbf{\phi},\mathcal{D}}(\boldsymbol{x},\boldsymbol{z}) = q_{\boldsymbol{\phi}}(\mathbf{z}\mid\mathbf{x})p_{\mathcal{D}}(\mathbf{x})
    $$
    where $p_{\mathcal{D}}(\mathbf{x})$ represents the true underlaying distribuion of data (however only represented by the samples).
    This is due to the direction of the KL divergence: The generative model is only slightly penalized when putting probability mass on values 
    of $(\boldsymbol{x},\boldsymbol{z})$ with no support under $q_{\mathbf{\phi},\mathcal{D}}$\cite{intro-vae-2019}. This corresponds to forward KL
    divergence visualized in (a) in \ref{fig:forward-reverse}.

    \item \textbf{Posterior collapse}: It has been observed, that part of the latent variable can collapsese during the training
    and become inactive \cite{lvae-2016,iaf-2016,bowmann-2016}, i.e. the optimization gets stuck in an undesirable stable
    equilibrium. This is due to the fact that objective $\mathrm{ELBO}$ is composed of the reconstruction error
    and the variational regularization term (see eq. \ref{eq:elbo_terms}). The reconustriction error is initially relatively week and 
    the variational term forces the approximate posterior towards its prior. This can result in many latent units collapsing before they 
    can learn a useful representation. 
    The \cite{lvae-2016} alleviate the problem by scheduling the optimization objective such that the variational regularization term is
    forced to 0 at the beginining of the optimization and then slowly linear ramped up to its value through the optimization. The \cite{iaf-2016} 
    proposes the method of \textit{free bits}, which ensures that on average, a certain minimum information is encoded per (group of) 
    latent variable. 
    \item \textbf{Disentangling latent factors}: Vannila VAEs often learn a tangled representation of the latent space, making it
    difficult to interpret or control individual factors in the data. This can limit their usefulness in applications such as 
    image editing or style transfer.
\end{itemize}

\section{Hiearchical VAE}
A hierarchical variational autoencoder (HVAE) is a generalization of the VAE that introduces multiple layers of latent variables.
The first proposal to address the limitations of the vanilla VAE was the use of auxiliary latent variables, as introduced in \cite{aux-var-2016},
which were followed up by truly hiearchical structure in \cite{lvae-2016} and \cite{iaf-2016}. Standard vanilla VAE contains only one layer
of latent variables (fig. \ref{fig:vanilla-vae}) in contrast to HVAE, such as markov HVAE, where each latent variable depends only on previous one
(fig. \ref{fig:mhvae}). The HVAE can be represented as (almost) any directed acyclic graph and can be represented as DLVM or Bayesian Network with 
following joint and posterior distribuions:
\begin{align}
    p_{\boldsymbol{\theta}}(\boldsymbol{x},\boldsymbol{z}_{1:T})&=p_{\boldsymbol{\theta}}(\boldsymbol{x}\mid\boldsymbol{z}_1) \prod_{t=1}^{T-1}\bigl[p_{\boldsymbol{\theta}}(\boldsymbol{z}_{t}\mid\boldsymbol{z}_{>t}) \bigr]  p_{\boldsymbol{\theta}}(\boldsymbol{z}_{T}) \label{eq:hvae_prior} \\
    q_{\boldsymbol{\phi}}(\boldsymbol{z}_{1:T}\mid\boldsymbol{x}) &= q_{\boldsymbol{\phi}}(\boldsymbol{z}_{1}\mid\boldsymbol{x}) \prod_{t=2}^{T}  q_{\boldsymbol{\phi}}(\boldsymbol{z}_{t}\mid\boldsymbol{z}_{<t}) \label{eq:hvae_posterior}
\end{align}
where $\boldsymbol{z}_{<t} = \boldsymbol{z}_{1:t-1}$ represents the sequence of all latent variables $\boldsymbol{z}_1$ up to $\boldsymbol{z}_t$. 
We can derive the ELBO objective for the HVAE: 
\begin{align}
    \log p_{\boldsymbol{\theta}}(\boldsymbol{x}) &= \log \int_{\boldsymbol{z}_{1:T}} p_{\boldsymbol{\theta}}(\boldsymbol{x},\boldsymbol{z}_{1:T}) \d \boldsymbol{z}_{1:T} \notag\\
    &= \log \int_{\boldsymbol{z}_{1:T}} \frac{p_{\boldsymbol{\theta}}(\boldsymbol{x}, \boldsymbol{z}_{1:T})q_{\boldsymbol{\phi}}(\boldsymbol{z}_{1:T}\mid \boldsymbol{x})}{q_{\boldsymbol{\phi}}(\boldsymbol{z}_{1:T}\mid\boldsymbol{x})} d\boldsymbol{z}_{1:T} \notag \\
    &= \log \EX_{q_{\boldsymbol{\phi}}(\boldsymbol{z}_{1:T}\mid\boldsymbol{x})} \left[\frac{p_{\boldsymbol{\theta}}(\boldsymbol{x}, \boldsymbol{z}_{1:T})}{q_{\boldsymbol{\phi}}(\boldsymbol{z}_{1:T}\mid\boldsymbol{x})}\right] \notag\\
    &\geq \mathbb{E}_{q_{\boldsymbol{\phi}}(\boldsymbol{z}_{1:T}\mid\boldsymbol{x})}\left[\log \frac{p_{\boldsymbol{\theta}}(\boldsymbol{x}, \boldsymbol{z}_{1:T})}{q_{\boldsymbol{\phi}}(\boldsymbol{z}_{1:T}\mid\boldsymbol{x})}\right]  = \mathrm{ELBO}(q_{\boldsymbol{\phi}}(\boldsymbol{z}_{1:T}\mid\boldsymbol{x})) \label{eq:hvae_elbo}
\end{align}
We can plug the joint and posterior distribuions from eq. \ref{eq:hvae_prior} and  eq. \ref{eq:hvae_posterior} into the ELBO term (eq. \ref{eq:hvae_elbo}) and obtain:

\begin{align}
    \mathrm{ELBO}(q_{\boldsymbol{\phi}}(\boldsymbol{z}_{1:T}\mid\boldsymbol{x}))  &=
    \mathbb{E}_{q_{\boldsymbol{\phi}}(\boldsymbol{z}_{1:T}\mid\boldsymbol{x})}\left[\log \frac{p_{\boldsymbol{\theta}}(\boldsymbol{z}_T)p_{\boldsymbol{\theta}}(\boldsymbol{x}\mid\boldsymbol{z}_1)\prod_{t=1}^{T-1}p_{\boldsymbol{\theta}}(\boldsymbol{z}_{t}\mid\boldsymbol{z}_{>t})}{q_{\boldsymbol{\phi}}(\boldsymbol{z}_1\mid\boldsymbol{x})\prod_{t=2}^{T}q_{\boldsymbol{\phi}}(\boldsymbol{z}_{t}\mid\boldsymbol{z}_{<t})}\right]
\end{align}

\begin{figure}[]
    \centering
    \begin{subfigure}[b]{0.27\textwidth}
        \centering
        \includegraphics[width=\textwidth]{vannila_vae.png}
        \caption[]{Vanilla VAE}
        \label{fig:vanilla-vae}
    \end{subfigure}
    \hfill
    \begin{subfigure}[b]{0.63\textwidth}
        \centering
        \includegraphics[width=\textwidth]{hvae.png}
        \caption[]{markov HVAE with $T$ layers, each varible $\boldsymbol{z}_i$ depends only on previous $\boldsymbol{z}_{i+1}$, creating a markov chain.}
        \label{fig:mhvae}
    \end{subfigure}
    \caption[VAE vs HVAE architecture]{Visualization of vanilla VAE and markov hiearchical VAE (MHVAE) architecture. Source \cite{diffusion-models-blog-2018}}
    \label{fig:vae-architecture}
\end{figure}


\subsection{Markov HVAE}
The special case of the HVAE is markov HVAE, where each varible $\boldsymbol{z}_i$ depends only on previous variable $\boldsymbol{z}_{i+1}$. This can be
seen as a multiple VAEs stacked on top of each other as represented in fig \ref{fig:mhvae}. The joint and posterior distribuion of HVAE is then
\begin{align*}
    p_{\boldsymbol{\theta}}(\boldsymbol{x},\boldsymbol{z}_{1:T})&=p_{\boldsymbol{\theta}}(\boldsymbol{x}\mid\boldsymbol{z}_1)\prod_{t=1}^{T-1} \bigl[   p_{\boldsymbol{\theta}}(\boldsymbol{z}_{t}\mid\boldsymbol{z}_{t+1}) \bigr]  p_{\boldsymbol{\theta}}(\boldsymbol{z}_{T}) \\
    q_{\boldsymbol{\phi}}(\boldsymbol{z}_{1:T}\mid\boldsymbol{x}) &= q_{\boldsymbol{\phi}}(\boldsymbol{z}_1\mid\boldsymbol{x})\prod_{t=2}^{T}q_{\boldsymbol{\phi}}(\boldsymbol{z}_{t}\mid\boldsymbol{z}_{t-1}) 
\end{align*}
with objective $\mathrm{ELBO}(q_{\boldsymbol{\phi}}(\boldsymbol{z}_{1:T}\mid\boldsymbol{x}))$:
\begin{align*}
    \mathbb{E}_{q_{\boldsymbol{\phi}}(\boldsymbol{z}_{1:T}\mid\boldsymbol{x})}\left[\log \frac{p_{\boldsymbol{\theta}}(\boldsymbol{z}_T)p_{\boldsymbol{\theta}}(\boldsymbol{x}\mid\boldsymbol{z}_1)\prod_{t=1}^{T-1}p_{\boldsymbol{\theta}}(\boldsymbol{z}_{t}\mid\boldsymbol{z}_{t+1})}{q_{\boldsymbol{\phi}}(\boldsymbol{z}_1\mid\boldsymbol{x})\prod_{t=2}^{T}q_{\boldsymbol{\phi}}(\boldsymbol{z}_{t}\mid\boldsymbol{z}_{t-1})}\right]
\end{align*}
While MHVAEs can express complex distributions and overcome the issue of blurriness, optimizing them for deep hierarchies can be challenging because of their multiple conditional stochastic layers.
This was partial overcomed with ladder VAE, which proposes different ordering of the variational encoder to improve the learning. 
\subsection{Ladder VAE}
The standard HVAEs (used in \cite{vae-original-2013,iaf-2016,renedze-backprop-2014}), which use a bottom-up encoder and top-down decoder, are
difficult to train due to the lack of interaction between the encoder and decoder during inference, as shown in (a) of fig. \ref{fig:lvae}. 
However, the recently proposed ladder VAE (LVAE) by Sønderby et al. \cite{lvae-2016} addresses this issue by introducing a new inference model
that leverages top-down dependencies, similar to the generative model, as depicted in (b) of fig. \ref{fig:lvae}. This approximate posterior 
distribution merges information from a bottom-up approximate likelihood and top-down prior information. By sharing information and parameters 
with the generative model, the inference model gains access to the current state of the generative model at each layer. The top-down pass then
recursively corrects the generative distribution with a data-dependent approximate log-likelihood.
\begin{figure}[t]
    \centering
    \includegraphics[width=0.55\textwidth]{lvae.png}
    \caption[Ladder VAE]{Ilustration of bottom-up and top-down approaches of encoder (inference model) and decoder (generative model). In (a), we can see the standard HVAE and in (b) the newly proposed LVAE.
    Source \cite{lvae-2016}}
    \label{fig:lvae}
\end{figure}
This inference model can be formaly expressed as:
\begin{equation}
q_{\boldsymbol{\phi},\boldsymbol{\theta}}(\boldsymbol{z}_{t}\mid\boldsymbol{z}_{>t}, \boldsymbol{x}) \sim p_{\boldsymbol{\theta}}(\boldsymbol{z}_{t}\mid\boldsymbol{z}_{>t})\tilde{q}_{\boldsymbol{\phi}}(\boldsymbol{z}_i \mid \boldsymbol{x}) \label{eq:lvae_encoder}
\end{equation}
where $\tilde{q}_{\boldsymbol{\phi}}(\boldsymbol{z}_i \mid \boldsymbol{x})$ represents the probability distribution from given family $\mathcal{Q}$ parametrized by (deterministic) encoder:
\begin{align*}
\boldsymbol{d}_i &= \mathrm{NeuralNet}_{\boldsymbol{\phi}} (\boldsymbol{d}_{i-1}) , \, \boldsymbol{d}_0 = \boldsymbol{x} \\
\tilde{q}_{\boldsymbol{\phi}}(\boldsymbol{z}_i \mid \boldsymbol{x}) &= \tilde{q}_{\boldsymbol{\phi}}(\boldsymbol{z}_i \mid \boldsymbol{d}_i )
\end{align*}
The multiplication in eq. \ref{eq:lvae_encoder} is a choice of authors, but is favorable in case the distribuions are members of exponential family (see \ref{sec:exp_family})













% Exp family 
\section{Exponential family}
The exponential family is a parametric set of probability distribuions, whose probability densities or masses can be expressed in form:
\begin{equation}
    p(\boldsymbol{x}\mid \boldsymbol{\eta}) = h(\boldsymbol{x}) \exp( \boldsymbol{T}(\boldsymbol{x}) \cdot \boldsymbol{\eta} - A(\boldsymbol{\eta})) \label{eq:exp-fam}
\end{equation}

where $h(\boldsymbol{x})$ is a base measure, $\boldsymbol{\eta}$ is vector of \textit{natural parameters}, 
$\boldsymbol{T}(\boldsymbol{x})$ are \textit{suffient statistics} and  $A(\boldsymbol{\eta})$ is \textit{cumulant function} 
also known as \textit{log normalizer} (see eq. \ref{eq:log_norm} for explanation).

Many common distributions, such as normal distribution, categorical distribution, 
Bernoulli distribuion, gamma distribution, Dirichlet distribution, etc belong to the exponential family. We show the reparametrization of some 
distribuions so they correspond to eq. \ref{eq:exp-fam} in table \ref{tab:exp-fam-reparametrization}
\begin{table}[ht]
    \centering
    \begin{tabular}{|c|c|c|c|c|c|c|}
      \hline
      Distribution & $\theta$ & $\eta$ & $h(x)$ & $T(x)$ & $A(\eta)$ & $A(\theta)$ \\
      \hline
      Bernoulli & $p$ & $\log \frac{p}{1-p}$ & $1$ & $x$ & $\log(1+e^{\eta})$ & $-\log(1-p)$\\
      \hline
      Categorical & $\begin{array}{c} p_1 \\ \vdots \\ p_k\end{array}$ & $\begin{bmatrix}\log p_1 \\ \vdots \\ \log p_k\end{bmatrix}$ & 1&  $\begin{bmatrix} [x=1] \\ \vdots \\ [x=k] \end{bmatrix}$ & 0& 0\\
      \hline
      Gaussian & $\begin{array}{c} \mu \\ \sigma^2 \end{array}$ & $\begin{bmatrix} \frac{\mu}{\sigma^2}\\ -\frac{1}{2\sigma^2} \end{bmatrix}$  & $\frac{1}{2\pi}$ & $\begin{bmatrix} x \\ x^2 \end{bmatrix} $  & $-\frac{\eta_1^2}{4\eta_2} - \frac{\log(-2\eta_2)}{2}$& $\frac{\mu^2}{2\sigma^2} + \log \sigma$\\
      \hline
    \end{tabular}
  \caption[Representatives of Exponential Family]{Some members of Exponential family. $\theta$ represents the standard parameter.Other symbols are described in eq. \ref{eq:exp-fam}.}
  \label{tab:exp-fam-reparametrization}
  \end{table}
  
\subsection*{Cumulant function}
Because the $p(\boldsymbol{x}\mid \boldsymbol{\eta})$ is a probability density, the integral of it equalls one:
\begin{align}
    \int_{\boldsymbol{x}} p(\boldsymbol{x}\mid \boldsymbol{\eta}) \d \boldsymbol{x} &= \int_{\boldsymbol{x}} h(\boldsymbol{x}) \exp( \boldsymbol{T}(\boldsymbol{x}) \cdot \boldsymbol{\eta} - A(\boldsymbol{\eta})) \d\boldsymbol{x} \notag  \\
    &=  \frac{ \int_{\boldsymbol{x}}  h(\boldsymbol{x})\exp( \boldsymbol{T}(\boldsymbol{x}) \cdot \boldsymbol{\eta})}{\exp(A(\boldsymbol{\eta}))}  \d\boldsymbol{x} = 1  \notag\\
    A(\boldsymbol{\eta}) &= \log\Bigl[ \int_{\boldsymbol{x}} h(\boldsymbol{x})\exp( \boldsymbol{T}(\boldsymbol{x}) \cdot \boldsymbol{\eta})\Bigr] \label{eq:log_norm}
\end{align}
and therefore the name \textit{log normalizer}. Another interesting property is that the derivative of cumulant function w.r.t. natural parameters is:
$$
\frac{d}{d \boldsymbol{\eta}} A(\boldsymbol{\eta}) = \EX_{\boldsymbol{x} \sim p(\boldsymbol{x}\mid \boldsymbol{\eta})} [\boldsymbol{T}(\boldsymbol{x})]
$$
this is easy to see since:
\begin{align*}
    \frac{d}{d \boldsymbol{\eta}} \int_{\boldsymbol{x}} p(\boldsymbol{x}\mid \boldsymbol{\eta}) \d \boldsymbol{x}&= \frac{d}{d \boldsymbol{\eta}}   \int_{\boldsymbol{x}} h(\boldsymbol{x}) \exp( \boldsymbol{T}(\boldsymbol{x}) \cdot \boldsymbol{\eta} - A(\boldsymbol{\eta})) \d\boldsymbol{x} \\
    &=\int_{\boldsymbol{x}}  \frac{\partial}{\partial \boldsymbol{\eta}} \bigl[h(\boldsymbol{x}) \exp( \boldsymbol{T}(\boldsymbol{x}) \cdot \boldsymbol{\eta} - A(\boldsymbol{\eta})) \d\boldsymbol{x} \bigr]  \\
    &=\int_{\boldsymbol{x}}  \bigl[h(\boldsymbol{x}) \exp( \boldsymbol{T}(\boldsymbol{x}) \cdot \boldsymbol{\eta} - A(\boldsymbol{\eta})) \d\boldsymbol{x} \bigr] \bigl[ \boldsymbol{T}(\boldsymbol{x}) - \frac{d}{d\boldsymbol{\eta}} A(\boldsymbol{\eta}) \bigr]\d\boldsymbol{x} \\
    &=\int_{\boldsymbol{x}}  \bigl[\boldsymbol{T}(\boldsymbol{x}) - \frac{d}{d\boldsymbol{\eta}} A(\boldsymbol{\eta})\bigr] p(\boldsymbol{x}\mid \boldsymbol{\eta}) \d\boldsymbol{x}  \\
    &=\EX_{\boldsymbol{x} \sim p(\boldsymbol{x}\mid \boldsymbol{\eta})}\bigl[\boldsymbol{T}(\boldsymbol{x}) - \frac{d}{d\boldsymbol{\eta}} A(\boldsymbol{\eta}) \bigr] = 0 \\
    \EX_{\boldsymbol{x} \sim p(\boldsymbol{x}\mid \boldsymbol{\eta})}\bigl[\boldsymbol{T}(\boldsymbol{x})\bigr] &= \EX_{\boldsymbol{x} \sim p(\boldsymbol{x}\mid \boldsymbol{\eta})} \bigl[\frac{d}{d\boldsymbol{\eta}} A(\boldsymbol{\eta}) \bigr] = \frac{d}{d\boldsymbol{\eta}} A(\boldsymbol{\eta}).
\end{align*}
The second derivative of cumulant function with respect to natural parameter is variance of sufficient statistic \cite{exp-family-jorden-2009}:
$$
\frac{d^2}{d \boldsymbol{\eta}^2} A(\boldsymbol{\eta}) = \mathbb{V}_{\boldsymbol{x} \sim p(\boldsymbol{x}\mid \boldsymbol{\eta})} [\boldsymbol{T}(\boldsymbol{x})]
$$
And then we also have a theorem about the convexity of the exponential family \cite{exp-family-jorden-2009}:
\begin{theorem}
The natural parameter space $\mathcal{N}$ is convex (as a set) and the cumulant function $A(\boldsymbol{\eta})$ is convex (as a function). 
If the family is minimal then $A(\boldsymbol{\eta})$ is strictly convex.
\end{theorem}
\subsection*{Exponential family and conjugate priors}
Within bayes network, we assume that the prior $\boldsymbol{\theta}$ is a random variable (r.v.) and thus we need to specify its \textit{prior distribution} $p(\boldsymbol{\theta})$.
We can obtain the \textit{posterior distribuion} via Bayes formula:
\begin{equation}
    p(\boldsymbol{\theta} \mid \boldsymbol{x}) = \frac{p( \boldsymbol{x} \mid \boldsymbol{\theta}) p(\boldsymbol{\theta})}{p(\boldsymbol{x})} \propto \underbrace{p( \boldsymbol{x} \mid \boldsymbol{\theta})}_{\text{likelihood}} \underbrace{p(\boldsymbol{\theta})}_{\text{prior}}
    \label{eq:bayes-formula}
\end{equation}
where 
\begin{equation}
    p(\boldsymbol{x}) = \int_{\boldsymbol{x}} p( \boldsymbol{x} \mid \boldsymbol{\theta}) p(\boldsymbol{\theta}) \d \boldsymbol{\theta} \label{eq:x-int}
\end{equation}
The idea of  \textit{conjugate prior} is following: Given a likelihood $p( \boldsymbol{x} \mid \boldsymbol{\theta})$, we choose a family of prior distribuions such that updating it through 
bayes formula (eq. \ref{eq:bayes-formula}) yeilds a posterior within same family as prior. Moreover the integrals of form eq. \ref{eq:x-int} has to be tractable. In general these two goals
are in conflict \cite{conjugates-jorden-2009}. 

We will now show that the for a likelihood from exponential family, there is conjugate prior from exponential family, i.e. for exponential family in canonical form: 
$$
h(\boldsymbol{x}) \exp( \boldsymbol{T}(\boldsymbol{x}) \cdot \boldsymbol{\eta} - A(\boldsymbol{\eta}))
$$
and random sample $\mathcal{D} = \{\boldsymbol{x}_1,\boldsymbol{x}_2\dots,\boldsymbol{x}_N\}$, we obtain the likelihood function:
$$
p(\mathcal{D}\mid \boldsymbol{\eta}) = \Bigl(\prod_{i=1}^N h(\boldsymbol{x}_n)\Bigr) \exp \Biggl\{ \biggl( \sum_{n=1}^N \boldsymbol{T}(\boldsymbol{x}) \biggr) \cdot \boldsymbol{\eta} - N A(\boldsymbol{\eta}) \Biggr\}
$$
we mimic the likelihood to obtain a probability density function:
$$ 
p(\boldsymbol{\eta} \mid \boldsymbol{\tau},n_0) = h^\prime(\boldsymbol{\eta}) \exp \Bigl\{ \boldsymbol{\eta} \cdot \boldsymbol{\tau} - n_0 A(\boldsymbol{\eta}) - A(\boldsymbol{\tau}, n_0)\Bigr\}
$$
we compute the posterior:
\begin{align*}
    p(\boldsymbol{\eta} \mid \mathcal{D},\boldsymbol{\tau},n_0) &\propto h^\prime(\boldsymbol{\eta})\Bigl(\prod_{i=1}^N h(\boldsymbol{x}_n)\Bigr) \exp \Biggl\{  \Bigl( \boldsymbol{\tau} +\sum_{n=1}^N \boldsymbol{T}(\boldsymbol{x}) \Bigr) \boldsymbol{\eta} - \bigl( n_0 + N\bigr)A(\boldsymbol{\eta}) - A(\boldsymbol{\tau}, n_0) \Biggr\}\\
 &\propto h^\prime(\boldsymbol{\eta}) \exp \Biggl\{  \Bigl( \boldsymbol{\tau} +\sum_{n=1}^N \boldsymbol{T}(\boldsymbol{x}) \Bigr) \boldsymbol{\eta} - \bigl( n_0 + N\bigr)A(\boldsymbol{\eta}) - \underbrace{A(\boldsymbol{\tau}, n_0) + \sum_{i=1}^{N} \log (h(\boldsymbol{x}_n))}_{\text{cumulant function}\,A(\mathcal{D},\boldsymbol{\tau},n_0)} \Biggr\}
\end{align*}
which is in the form of exponential family. The update rules are
\begin{align*}
\boldsymbol{\tau} &\mapsto \boldsymbol{\tau}  +\sum_{n=1}^N \boldsymbol{T}(\boldsymbol{x}) \\
n_0 &\mapsto n_0 + N 
\end{align*}
\subsection*{Kullback Leibler divergence}
The KL divergence for two distribuions $p$ and $q$ is defined as:
$$
\mathrm{KL}(p(\boldsymbol{x}) || q(\boldsymbol{x})) = \int_x p(\boldsymbol{x}) \log \frac{p(\boldsymbol{x})}{q(\boldsymbol{x})} \d \boldsymbol{x} 
= \EX_{p(\boldsymbol{x})} \log \frac{p(\boldsymbol{x})}{q(\boldsymbol{x})} 
$$
however for the distribuions of the family, one can obtain a closed formula:
\begin{align*}
    \mathrm{KL}(p(\boldsymbol{x}) || q(\boldsymbol{x})) &= \EX_{p(\boldsymbol{x})} (\boldsymbol{\eta}_p - \boldsymbol{\eta}_q) \cdot  \boldsymbol{T}(\boldsymbol{x}) - A(\boldsymbol{\eta}_p) + A(\boldsymbol{\eta}_q) \\
    &= (\boldsymbol{\eta}_p - \boldsymbol{\eta}_q) \cdot  \boldsymbol{\mu}_p - A(\boldsymbol{\eta}_p) + A(\boldsymbol{\eta}_q) \\
\end{align*}
where $\mu_p =\EX_{p(\boldsymbol{x})}[\boldsymbol{T}(\boldsymbol{x})] $ is the mean parameter and can be obtained through differentiating the cumulant function. 

We define the \textit{empirical data distribution}
$$
p_{\mathcal{D}} = \frac{1}{|\mathcal{D}|} \sum_{\boldsymbol{x}^\prime \in \mathcal{D}} \delta(\boldsymbol{x},\boldsymbol{x}^\prime)
$$
where $\delta(\boldsymbol{x},\boldsymbol{x}^\prime)$is Kronecker delta.  This distribuion places a point mass at each datapoint in dataset $\mathcal{D}$.
We can utilize it for writting the log likelihood (in discrete case):
\begin{align*}
\sum_{\boldsymbol{x}} p_{\mathcal{D}} \log p(\boldsymbol{x} \mid \boldsymbol{\theta}) &= \sum_{\boldsymbol{x}} \frac{1}{|\mathcal{D}|} \sum_{\boldsymbol{x}^\prime \in \mathcal{D}} \delta(\boldsymbol{x},\boldsymbol{x}^\prime) \log  p(\boldsymbol{x} | \boldsymbol{\theta}) \\
    &= \frac{1}{|\mathcal{D}|} \sum_{\boldsymbol{x}^\prime \in \mathcal{D}} \sum_{\boldsymbol{x}} \delta(\boldsymbol{x},\boldsymbol{x}^\prime) \log  p(\boldsymbol{x} | \boldsymbol{\theta})\\
    &= \frac{1}{|\mathcal{D}|} \sum_{\boldsymbol{x}^\prime \in \mathcal{D}} \log  p(\boldsymbol{x}^\prime | \boldsymbol{\theta}) \\
    &= \frac{1}{|\mathcal{D}|} l(\boldsymbol{\theta} \mid \mathcal{D})
\end{align*}
where $l(\boldsymbol{\theta} \mid \mathcal{D}) = \log p(\mathcal{D} | \boldsymbol{\theta})$ is the log likelihood. So computing the cross entropy between empirical data distribution and model provides us with log likelihood.
If we compute the KL divergence of the empirical data ditribution and model $p(\boldsymbol{x} | \boldsymbol{\theta})$, we obtain
$$
\mathrm{KL}(p_{\mathcal{D}} || p(\boldsymbol{x} | \boldsymbol{\theta})) = \sum_{\boldsymbol{x}} p_{\mathcal{D}} \frac{p_{\mathcal{D}}}{p(\boldsymbol{x} | \boldsymbol{\theta})} = \EX_{p_{\mathcal{D}}} \log p_{\mathcal{D}} - \frac{1}{N} l(\boldsymbol{\theta} \mid \mathcal{D})
$$ 
the empirical data distribution is not depending on the model parameters $\boldsymbol{\theta}$ and thus by minimizing the KL divergence to the empirical distribuion, we maximize the (log) likelihood. 

% Symmetrical learning paper
\section{Symmetric leaning in VAE}
\label{sec:symmetric_learning}
The authors of~\cite{sym-learning-2023} present an alternative approach to maximizing the evidence lower bound (ELBO)
in Variational Autoencoders (VAEs). The traditional ELBO optimization imposes restrictions on the architectures of 
VAEs, as it requires the latent distributions to be in closed form while only providing data samples. This asymmetry in
the ELBO formulation contributes to the issue of blurriness in generated images (discussed in~\ref{item:blurriness-of-img}),
which has been partially addressed by methods like normalizing flows~\cite{nf-2015} and LVAE~\cite{lvae-2016}.

The proposed symmetric learning approach relaxes these restrictions and enables VAE learning when both the data and 
latent distributions are accessible only through sampling. This approach is also applicable to more complex models, such 
as Hierarchical VAEs (HVAEs), and leads to simpler algorithms for training. The conducted experiments show that models obtained 
from this training approach are comparable to those achieved through ELBO learning.

In standard VAE framework, we train the encoder  and decoder through maximizing the ELBO objective, i.e.
given the true underlying distribuion of data $pi(\boldsymbol{x}), \boldsymbol{x} \in \mathcal{X}$ and underlaying
the distribuion in latent variable $\pi(\boldsymbol{z}), \boldsymbol{z} \in \mathcal{Z}$, we maximize ELBO:
$$
\mathcal{L}_{B} = \mathrm{ELBO} = \EX_{\pi(\boldsymbol{x})} \bigl[ \EX_{q_{\boldsymbol{\phi}}(\boldsymbol{z} \mid \boldsymbol{x})} 
\log p_{\boldsymbol{\theta}}(\boldsymbol{x} \mid \boldsymbol{z}) - \mathrm{KL} (q_{\boldsymbol{\phi}}(\boldsymbol{z} \mid \boldsymbol{x}) || 
\pi(\boldsymbol{z})) \bigr]
$$
in order to obtain the pair of encoder $q_{\boldsymbol{\phi}}(\boldsymbol{z} \mid \boldsymbol{x})$ and decoder
$p_{\boldsymbol{\theta}}(\boldsymbol{x} \mid \boldsymbol{z})$. To keep the computation of KL divergence tractable, it is necessary to define
the model distribuion $p(\boldsymbol{z})$ in closed form. This is issue in case the $\pi(\boldsymbol{z})$ is complex and we cannot model 
it by a simple distribuion family. Another necessity is that the $q_{\boldsymbol{\phi}}(\boldsymbol{z} \mid \boldsymbol{x})$ allow the 
\textit{reparametrization trick}. 

The authors propose a new algotihm for learning the encoder $q_{\boldsymbol{\phi}}(\boldsymbol{z} \mid \boldsymbol{x})$ and 
decoder $p_{\boldsymbol{\theta}}(\boldsymbol{x} \mid \boldsymbol{z})$ in case of \textit{semi-supervised} and 
\textit{unsupervised} learning:
\begin{itemize}
    \item \textit{Semi-supervised learning}: We can draw i.i.d samples from underlaying unknown distribuions $(\boldsymbol{x},\boldsymbol{z})  \sim \pi(\boldsymbol{x},\boldsymbol{z})$
    and its marginals: $\boldsymbol{x} \sim \pi(\boldsymbol{x})$, $\boldsymbol{z} \sim \pi(\boldsymbol{z})$.
    \item \textit{Unsupervised learning}: We can draw only $\boldsymbol{x} \sim \pi(\boldsymbol{x})$. The latent space is modeled
    through the choice of model $p(\boldsymbol{z})$.
\end{itemize}

The encoder and decoder belong to the exponential family and allow for tractable computation of log density and its derivatives.
\begin{align*}
    p_{\boldsymbol{\theta}}(\boldsymbol{x} \mid \boldsymbol{z}) &\propto \exp \bigl[ \boldsymbol{\Theta}(\boldsymbol{x}) \cdot \boldsymbol{f}_{\boldsymbol{\theta}}(\boldsymbol{z}) \bigr] \\
    p_{\boldsymbol{\phi}}(\boldsymbol{x} \mid \boldsymbol{z}) &\propto \exp \bigl[ \boldsymbol{\Phi}(\boldsymbol{z}) \cdot \boldsymbol{g}_{\boldsymbol{\phi}}(\boldsymbol{x}) \bigr]
\end{align*} 
where $\boldsymbol{\Theta}(\boldsymbol{x}) \in \mathbb{R}^n$ and $ \boldsymbol{\Phi}(\boldsymbol{z}) \in \mathbb{R}^m$ are sufficient
statistics. The variables $\boldsymbol{x}$ and $\boldsymbol{z}$ can be either discrete or continous depending on the choice of exponential family (e.g. Bernoulli or Gaussian). 

The authors provide new optimization function, which is motivated through finding a \textit{Nash equilibrium} for two-player game 
where players' strategies are represented through the encoder and decoder distributions respectively and utility function is
a sum of the player expectation w.r.t his strategy~\cite{sym-learning-2023}. 
The objectives are 
\begin{align*}
    \mathcal{L}_{p}(\boldsymbol{\theta},\boldsymbol{\phi}) &= \EX_{\pi(\boldsymbol{x},\boldsymbol{z})}\bigl[\log p_{\boldsymbol{\theta}}(\boldsymbol{x},\boldsymbol{z})\bigr]+
                    \EX_{\pi(\boldsymbol{z})}\bigl[\log p_{\boldsymbol{\theta}}(\boldsymbol{z})\bigr] +
                    \EX_{\pi(\boldsymbol{x})}\EX_{q_{\boldsymbol{\phi}}(\boldsymbol{z} \mid\boldsymbol{x})}\bigl[\log p_{\boldsymbol{\theta}}(\boldsymbol{x},\boldsymbol{z})\bigr]\\
    \mathcal{L}_{q}(\boldsymbol{\theta},\boldsymbol{\phi}) &= \EX_{\pi(\boldsymbol{x},\boldsymbol{z})}\bigl[\log q_{\boldsymbol{\phi}}(\boldsymbol{z} \mid \boldsymbol{x})\bigr] + 
    \EX_{\pi(\boldsymbol{z})}\EX_{p_{\boldsymbol{\theta}}(\boldsymbol{x} \mid\boldsymbol{z})} \bigl[\log q_{\boldsymbol{\phi}}(\boldsymbol{z} \mid \boldsymbol{x})\bigr]
\end{align*}
for semi-supervised training and 
\begin{align}
    \mathcal{L}_{p}(\boldsymbol{\theta},\boldsymbol{\phi}) &= \EX_{\pi(\boldsymbol{x})}\EX_{q_{\boldsymbol{\phi}}(\boldsymbol{z} \mid\boldsymbol{x})}\bigl[\log p_{\boldsymbol{\theta}}(\boldsymbol{x},\boldsymbol{z})\bigr] \label{eq:unsup_objective_theta}\\
    \mathcal{L}_{q}(\boldsymbol{\theta},\boldsymbol{\phi}) &= \EX_{p_{\boldsymbol{\theta}}(\boldsymbol{x},\boldsymbol{z})} \bigl[\log q_{\boldsymbol{\phi}}(\boldsymbol{z} \mid \boldsymbol{x})\bigr] \label{eq:unsup_objective_phi}
\end{align}
for unsupervised training with following interpretation: We maximize the decoder likelihood and encoder likelihood of the training data at same time. The mixed 
terms reinforce the encoder decoder consistency. This corresponds to maximization of the ELBO objective, since we can 
rewrite the ELBO into:
$$
\EX_{\pi(\boldsymbol{x})} \bigl[ 
\log p_{\boldsymbol{\theta}}(\boldsymbol{x}) - \mathrm{KL} (q_{\boldsymbol{\phi}}(\boldsymbol{z} \mid \boldsymbol{x}) || 
p_{\boldsymbol{\theta}}(\boldsymbol{z} \mid \boldsymbol{x})) \bigr]
$$
After inspecting the terms, we see that the ELBO goal is same as above: To maximize the data likelihood and reinforce the consistency of the decoder-encoder pair 
simultaneously. 
\subsection{Hiearchical VAEs}
The algorithm can be also adopted for hiearchical VAE. Let us assume that we have HVAE with $M+1$ layers, i.e. $\boldsymbol{z}$ consists of 
$\boldsymbol{z}_0,\boldsymbol{z}_1,\dots,\boldsymbol{z}_m$, where the encoder models corresponds to the LVAE and we can sample $\boldsymbol{x} \sim \pi(\boldsymbol{x})$.
The encoder and decoder factorizes (the ordering is in reverse to the one in eq.~\ref{eq:hvae_prior} and eq.~\ref{eq:hvae_posterior}):
\begin{align*}
    p_{\boldsymbol{\theta}}(\boldsymbol{x},\boldsymbol{z})&=p_{\boldsymbol{\theta}}(\boldsymbol{z}_{0}) \prod_{t=1}^{M}\bigl[p_{\boldsymbol{\theta}}(\boldsymbol{z}_{t}\mid\boldsymbol{z}_{<t}) \bigr]  p_{\boldsymbol{\theta}}(\boldsymbol{x}\mid\boldsymbol{z})  \\
    q_{\boldsymbol{\phi}}(\boldsymbol{x},\boldsymbol{z}) &=\pi(\boldsymbol{x})q_{\boldsymbol{\phi}}(\boldsymbol{z}_{0}\mid\boldsymbol{x}) \prod_{t=1}^{M}  q_{\boldsymbol{\phi}}(\boldsymbol{z}_{t}\mid\boldsymbol{z}_{<t},\boldsymbol{x}) 
\end{align*}
where the encoder shares the parameter as described in~\ref{eq:lvae_encoder}. The objectives remain as in unsupervised case (eq.~\ref{eq:unsup_objective_theta} and eq.~\ref{eq:unsup_objective_phi}).
The terms can be decomposed due to the factorization of decoder and encoder and are thus tractable.  If there is 
acess to the samples $(\boldsymbol{x},\boldsymbol{z}_0) \sim \pi(\boldsymbol{x},\boldsymbol{z}_0)$,e.g. segmentation task with target masks, we can utilize them 
by addding terms
$$
\EX_{\pi(\boldsymbol{x},\boldsymbol{z}_0)}\EX_{q(\boldsymbol{z}_{>0} \mid\boldsymbol{z}_0,\boldsymbol{x})} \log p_{\_\theta}(\boldsymbol{x},\boldsymbol{z}) 
\quad\text{and}\quad
\EX_{\pi(\boldsymbol{x},\boldsymbol{z}_0)} \log q_{\_\phi}(\boldsymbol{z}_0\mid\boldsymbol{x})
$$
to the decoder and encoder respectively. 










\chapter{Methods}
\label{methods}
% Metics, dataset, task etc.
\section{Problem definition}
Our main objective is to develop an effective approach for segmenting satellite imagery using a limited number of 
accessible images and a smaller number of expert-provided annotations. The segmentation task involves assigning a 
specific label to each pixel in the image, effectively partitioning the image into distinct regions based on their 
semantic or visual characteristics.

To achieve this objective, we will employ the U-Net architecture, which has been proven to be effective for image 
segmentation~\cite{unet-2015}. The U-Net architecture will serve as the core model for evaluating the proposed algorithms.
We will use the plain accuracy and average ``intersection over union'' (IoU) metrics to evaluate the 
quality of the segmentation. The IoU measures the overlap between the predicted segmentation and the ground truth 
segmentation, i.e.:
$$
\mathrm{IoU} = \frac{\text{area of intersection}}{\text{area of union}},
$$
where the area of intersection corresponds to the area where the model predicts the given class and the class is in the ground truth segmentation.
The union area is then composed of an area where either model predicts the class or the class is in the ground truth segmentation (or both).
This can be rephrased to binary classification terminology as: 
$$
\mathrm{IoU} = \frac{\mathrm{TP}}{\mathrm{TP}+\mathrm{FP}+\mathrm{FN}},
$$
where the $\mathrm{TP}$ is \textit{true positive}, $\mathrm{FP}$ is \textit{false positive} and $\mathrm{FN}$ is \textit{false negative}. 
The average IoU is then the simple average of the class IoU over all classes. 

To evaluate our methods and ensure reproducible experiments, we will utilize the CityScape dataset~\cite{cityscapes-2016} as a benchmark. The CityScape dataset is almost publicly available, with a fee-less 
registration required to access. Although the dataset primarily consists of urban street scenes, we believe that the complexity of segmentation in this dataset is comparable to, or even higher than, the 
challenges encountered in land-coverage segmentation of forests.

By leveraging the CityScape dataset and employing the U-Net architecture, we can thoroughly evaluate our methods and 
provide meaningful results that others can replicate and validate.

The U-Net is a feed-forward convolutional neural network (FF CNN), and its architecture consists of an encoder and a decoder part, which work together for image segmentation tasks. 
(We would like to point out to the reader that in this context, the  ``encoder'' and ``decoder'' are not referring to the VAE networks but to the structure of the U-Net architecture):
\begin{itemize}
    \item \textit{Encoder}: The encoder part of U-Net is responsible for downscaling the spatial resolution of the input
     image while increasing its channel capacity. This is achieved through a series of convolutional blocks and downsampling. 
     Each block typically consists of one or more convolutional layers, followed by non-linear activation
    functions (such as ReLU) and batch normalization. The encoder acts as a feature extractor for image
    classification tasks.
    \item \textit{Decoder}: The decoder part of U-Net is connected to the encoder and performs the opposite operation
     of the encoder. It upscales the feature maps while reducing the number of channels. This is done through 
     a series of upsampling and convolutional blocks. Like the encoder, each block consists of convolutional
    layers, activation functions, and batch normalization. 

    \item \textit{Skip Connections}: U-Net incorporates skip connections between the encoder and decoder. These 
    connections allow information from the encoder to be directly propagated to the corresponding decoder 
    block at the same spatial resolution. By sharing this information, U-Net helps to preserve spatial details
     and enables more precise segmentation.

    \item \textit{Convolutional Blocks}: The convolutional blocks in U-Net typically consist of convolutional layers, activation 
    functions, and batch normalization. The number of convolutional layers in each block can vary but is usually up to 2 or 3. 
    These blocks are responsible for learning and extracting features from the input data. Their parameters are typically chosen such
    that they retain the spatial resolution

    \item \textit{Spatial Resolution}: U-Net can maintain the spatial resolution of the input or perform downsampling and upsampling 
    operations at specific layers, depending on the chosen parameters. This flexibility allows U-Net to
     capture both local and global information.
\end{itemize}
Overall, the U-Net architecture is designed to effectively capture contextual information and spatial details for image segmentation tasks. It has been widely used and has shown promising results in various applications.
% Mixmatch adaptation
\section{Mixmatch adaptation}
The current implementation of MixMatch algorithm is limited as it only allows for augmentations that do not alter the labels of given features. 
However, this limitation poses a problem for image segmentation tasks because most commonly used augmentations do apply spatial transformations 
that affect image segmentations. To address this issue, we propose an extension of the MixMatch algorithm specifically designed for image s
egmentation tasks.

Our approach involves adapting the data augmentation and proxy-labeling procedure to allow for augmentations that modify the corresponding 
segmentation masks. Specifically, we suggest using affine transformations for the data augmentations, as they are easily invertible and widely
applicable.

To combine predictions across multiple views of an image, we align the predictions with the original image by applying the inverse augmentation
and then apply the original augmentation to assign the averaged predictions to each view. However, it's important to note that care must be taken
while averaging because a part of the original image may be cropped while keeping the spatial dimensions. Thus, the segmentation prediction is 
only valid on the uncropped region of the image.

Finally, the MixUp procedure requires adaptation to ensure that cropped images are correctly mixed without propagating
the empty parts further down the stream while keeping the output of MixMatch ($\mathbf{x}^\prime,\mathbf{p}^\prime$) 
closer to the first argument $(\mathbf{x}^1,\mathbf{p}^1)$.

The proposed changes have been incorporated into algorithm \ref{alg:mixmatch-seg} and are commented there. In MixUp, we 
ensure that the cropped out parts of the images are suppressed so that they do not influence the MixUp process. 
Additionally, we aim to keep the output image closer to the first argument. The modified MixUp procedure is as follows
\begin{align}
  &\lambda \sim \text{Beta}(\alpha,\alpha) \notag\\
  &\lambda^\prime = \max(\lambda,1-\lambda) \notag\\
  &\_\lambda^\prime_{i,j} =\begin{cases}
    \lambda^\prime & \text{if } \_x^{1}_{i,j} \text{ and }  \_x^{2}_{i,j} \text{ are valid}\\
    1 & \text{if }  \_x^{1}_{i,j} \text{ is not valid or }  \_x^{2}_{i,j} \text{ is not valid}\\
  \end{cases} \label{eq:mixup_alternated}\\
  &\_x^\prime = \_\lambda^\prime \odot \_x^1 + (\_1-\_\lambda^\prime) \odot \_x^2 \notag\\
  &p^\prime = \_\lambda^\prime \odot \_p^1 + (\_1-\_\lambda^\prime) \odot \_p^2. \notag
\end{align}
Here, $\odot$ represents element-wise multiplication, and the upper index is used for indexing. The alternation
 of $\_{\lambda}^\prime$ ignores the invalid parts of the second input and retains the original picture 
 with its invalid parts. This modification aims to keep the output of the MixMatch procedure close to the input, as 
 we have inputs of uneven quality, i.e., labeled and unlabeled data.

\begin{algorithm}[t]
    \caption{MixMatch adapted for segmentation}
    \label{alg:mixmatch-seg}
    \begin{algorithmic}[1]
      \State \textbf{Input:} Batch of labeled examples and their segmentation masks $X = ((\_{x}_i, \_{p}_i);\,i \in (1, \dots, n))$, batch of unlabeled examples $U = (\_{u}_i; i \in (1, \dots, n))$, sharpening temperature $T$, number of augmentations $K$, Beta distribution parameter $\alpha$ for MixUp.
      \For{$i = 1$ \textbf{to} $n$}
       \State $\bar{\_x}_i,\bar{\_p}_i = \text{Augment}(\_x_i,\_p_i)$ \Comment{We apply augmentation both to $\_x_i$ and $\_p_i$}
       \For{$k = 1$ \textbf{to} $K$}
        \State $\bar{\_u}_{i,k} = \text{Augment}_k(\_u_i)$ 
        \State $\bar{\_q}_{i,k} = p_{\text{model}}(y|\bar{u}_{i,k}; \theta)$
       \EndFor
       \State $\tilde{\_q}_{i,k} = \text{Inverse Augment}_k(\bar{\_q}_{i,k})$ \Comment{Align prediction with original image}
       \State $\bar{\_q}_i = \frac{1}{K} \sum_{k=1}^{K} \tilde{\_q}_{i,k} $
       \State $\_q_i = \text{Sharpen}(\bar{\_q}_i, T)$ 
       \For{$k = 1$ \textbf{to} $K$}
        \State $\_q_{i,k} = \text{Augment}_k(\_q_i)$ \Comment{Rematch the average prediction to augmented image}
       \EndFor
      \EndFor
      \State $X^\star = ((\bar{\_x}_i, \_p_i); i \in (1, \dots, n))$ 
      \State $U^\star = ((\bar{\_u}_{i,k}, \_q_{i,k}); i \in (1, \dots, n), k \in (1, \dots, K))$ 
      \State $W = \text{Shuffle}(\text{Concat}(X^\star, U^\star))$ 
      \State $X^\prime = (\text{MixUp}(\bar{\_x}_i, \_w_i); i \in (1, \dots, |X^\star|))$ \Comment{Apply MixUp described in \ref{eq:mixup_alternated}}
      \State $U^\prime = (\text{MixUp}(\bar{\_u}_{i}, \_w_{i+|X^\star|}); i \in (1, \dots, |U^\star|))$ 
      \State \textbf{return} $X^\prime, U^\prime$
      
    \end{algorithmic}
   \end{algorithm}
% HVAE design and choice 
\section{Symetric learning for HVAE}
We implement a hierarchical variational autoencoder (HVAE) that aims to provide a segmentation 
$\mathbf{s}$ for a given image $\mathbf{x}$ and reconstruct the image if a ground truth segmentation is provided. 
In our scenario, we only have access to samples from the unknown underlying distribution of images and
segmentations, denoted as $\mathbf{x}, \mathbf{s} \sim \pi(\mathbf{x},\mathbf{s})$.

The hierarchical VAE consists of $M+1$ layers of latent space:
\begin{align*}
\mathbf{z} = (\mathbf{z}_0, \mathbf{z}_1, \dots, \mathbf{z}_m), \quad \mathbf{z}_0 = (\mathbf{s}, \mathbf{l}),\, \mathbf{z}_m = \mathbf{x}
\end{align*}
Here, $\mathbf{z}_m$ corresponds to the image, and $\mathbf{z}_0$ is the composition of the segmentation $\mathbf{s}$ and 
the latent code $\mathbf{l}$, which follows a uniform prior distribution. The latent code is expected to encode global
information in the image (e.g., weather, texture), while the segmentation provides local information (e.g., road shape,
surrounding types, pedestrians, cars, etc.). The hierarchical model consists of an encoder and
 a decoder with a common factorization:
\begin{align*}
    p_{\boldsymbol{\theta}}(\boldsymbol{x},\boldsymbol{z})&=p_{\boldsymbol{\theta}}(\boldsymbol{z}_{0}) \prod_{t=1}^{M}\bigl[p_{\boldsymbol{\theta}}(\boldsymbol{z}_{t}\mid\boldsymbol{z}_{<t}) \bigr]  p_{\boldsymbol{\theta}}(\boldsymbol{x}\mid\boldsymbol{z})  \\
    q_{\boldsymbol{\phi}}(\boldsymbol{x},\boldsymbol{z}) &=\pi(\boldsymbol{x})q_{\boldsymbol{\phi}}(\boldsymbol{z}_{0}\mid\boldsymbol{x}) \prod_{t=1}^{M}  q_{\boldsymbol{\phi}}(\boldsymbol{z}_{t}\mid\boldsymbol{z}_{<t},\boldsymbol{x}) 
\end{align*}
Both the encoder $q_{\boldsymbol{\phi}}$ and decoder $p_{\boldsymbol{\theta}}$ have a U-Net-like architecture. Since
 the U-Net is a feedforward network, sampling from the decoder is tractable and we truly can construct the HVAE. 
 The conditional models in the HVAE belong to different distributions within the exponential family, depending on the 
 specific case. The hidden layers $\mathbf{z}_1, \dots, \mathbf{z}_{m-1}$, as well as the latent 
 code $\mathbf{l}$, are modeled using a Bernoulli distribution. For the image and segmentation, we utilize Gaussian
 and categorical distributions, respectively.
 

We apply the standard construction blocks of the U-net, up to the last (first) layer of encoder (decoder), where we 
face a challenge when combining the segmentation $\mathbf{s}$ and latent code $\mathbf{l}$ since the latent code does
not possess spatial dimensions.
\begin{itemize}
    \item The \textit{encoder} consists of two independent networks, meaning they do not share parameters. 
    The first network is a standard convolutional block responsible for mapping the input to segmentation activations. 
    The number of output channels in this block corresponds to the number of classes in the segmentation task. 
    The second network is an extended convolutional block that incorporates an adaptive average pooling layer. 
    This layer reduces the spatial dimensions to 1, generating the latent code activations.

    \item The \textit{decoder} part comprises a single convolutional block. Initially, we replicate the latent
    code activations in the spatial dimension to match the segmentation's spatial dimensions. Next, we concatenate
    the replicated latent code and the segmentations along the channel dimension, creating a compact block. This
    block serves as the input to the aforementioned convolutional block in the decoder.
\end{itemize}
By replicating the latent code in the spatial dimension and combining it with the segmentations, we ensure that the
latent code $\mathbf{l}$ possesses complete information about the image and has a global influence. 





\chapter{Experiments \& Results} 
\label{results}
\section{Mixmatch experiments}
As the original implementation of MixMatch is written in \texttt{TensorFlow} we have re-implemented it in \texttt{PyTorch}. To verify the correctness of our implementation, we have run tests on CIFAR10 and compared our results with those reported in \cite{mixmatch-2019} and \cite{wide-resnet-2017}. The 
comparison is shown in table \ref{tab:mixmatch-cifar10}. For details of reported results, see respecitve papers. 
The specific hyperparameters of experiments are provided with the code and are available. We will not report them here.
 
\begin{table}[htb]
    \begin{tabular}{|c|c|c|c|c|c|c|}
    \hline
    Labels [\#]  & 250 & 500 & 1000 & 2000 & 4000 & All \\
    \hline
    Our code & $88.45$ & $89.58$& $91.83$ & $93.03$ & $93.50$ & $93.54$\\
    \hline
    Reported & $88.92\pm0.87$ & $90.35\pm0.94$ & $92.25\pm0.32$ & $92.97\pm0.15$ & $93.76\pm 0.06$ & $94.27$\\
    \hline 
    Baseline & $38.42$ & $45.77$ & $50.42$ & $60.21$ & $79.57$ &  \\
    \hline
    \end{tabular}
    \caption[Mixmatch accuracy on CIFAR10]{Mixmatch accuracy rate (\%) on CIFAR10 dataset \cite{cifar10-2009}. Labels row corresponds to number of labeled points available during trainig. 
    The last column ("All") corresponds to fully-supervised mode performance.
    We additionally provide the supervised row, which contains the supervised baseline trained on given number of images (Our code).
    Our results are based only on one run.}
    \label{tab:mixmatch-cifar10}
\end{table}

We have 
\todo{Write about the selection of the categories, etc.}
     

\todo{Presentation of the results of the experiments}
\todo{Comparison of the results of the ladder autovariational encoder with symmetrical learning and mixmatch models}
\todo{Discussion of the results and analysis of the findings}



\chapter{Conclusion}
\label{conclusions}
\todo{2-3 pages}	
\todo{Summary of the main findings and contributions}

\printbibliography

%\ctutemplate{specification.as.chapter}

\chapter*{List of used symbols, notations}
% \noindent
% \todo{add rest, verify all typesetting}
% \begin{tabularx}{\linewidth}
% { l >{\raggedright\arraybackslash}X }
% \bfseries Symbols & \bfseries Description \\\Midrule
% $\mathbf{x},\mathbf{y},\mathbf{z}$ & vectors, collections of random variables \\
% $x,y,z$  & scalars \\
% $H,E$    & random variables \\
% $ \EX_q, \EX_{\mathbf{z}\sim q(\mathbf{z})}$ & expectation over distribution $q(\mathbf{z})$ \\
% $H(p)$ & (information) entropy \\
% $\mathbb{N}$ & Natural numbers \\
% $\mathbb{R}$ & Real numbers \\
% \end{tabularx}

% \begin{tabularx}{\linewidth}
% { l >{\raggedright\arraybackslash}X }
% \bfseries Zkratka & \bfseries Význam \\\Midrule
% $C$-space & konfigurační prostor \\
% NP  & nedeterministicky polynomiální (třída složitosti) \\
% AFP & artifical potential field \\
% PRM & probabilistic road map \\
% RRT & rapidly exploring random trees \\
% S\&C lattice & state and control lattice 
%\end{tabularx}


\end{document}