% Background (1-2 pages)

% Introduction to the topic and its importance
% Explanation of the real-world problem or scenario that the thesis aims to address
% Overview of the current state of research and any recent advances in the field
% Discussion of any challenges or limitations of existing approaches to the problem or scenario

Satellite imagery provides a wealth of information about our planet and has become a standard tool for monitoring, predicting and understanding 
the change in vegetation, agriculture and human environmental impact. With advancements in satellite technology, it is now possible to collect
considerably large amounts of high-resolution multispectral images over time, enabling researchers to perform complex spatiotemporal analyses of these 
datasets. However, the magnitude of the raw data makes it challenging to process and analyze effectively. Moreover the remote sensing 
still faces other challenges, which are rare in other areas of computer vision. Those are among others partial measurements 
(i.e. cloud cover, electro-magnetic (EM) interference), geolocation, different quality of measurement (spatial resolution, different EM bands) and 
calibration (issue of atmospheric reflectance).

From a machine learning perspective, the main challenges are the quite large amounts of unannotated data and partially missing measurements.
To adress the first, semi-supervised learning \cite{ssl-book-2006} (SSL) has shown promise in leveraging the abundance of unlabeled
data to improve model performance. SSL algorithms aim to learn from labeled and unlabeled data, using the labeled data to guide the 
learning process and the unlabeled data to regularize the model. Regarding the issue of partially missing measurements, generative
models, such as variational autoencoders (VAEs) \cite{intro-vae-2019}, have shown the potential to fill in missing data gaps. 
By learning a generative model of the data distribution, VAEs can be used to impute missing values in a dataset, making it possible to 
fully utilize the available data.

The thesis topic is motivated by the real-world problem of segmenting national park's satellite imagery to monitor and predict 
its forest development. The forest development prediction could enable national park rangers to respond proactively to protect forest vegetation and thus
improve the national park protection. The thesis aims to create an approach that could be used in preprocessing to obtain segmentation for many images and 
generate inpainting for partially missing measurements. The project is funded by European Space Agency (ESA)
\footnote[1]{\url{https://eo4society.esa.int/projects/spatiotemporal-sen2vhr/}}.

The authors of \cite{sym-learning-2023} proposed a novel SSL algorithm that employs symmetrical learning with hierarchical VAEs, which is capable of
handling exponential families of distribution, not just multivariate Gaussians for the latent variable. We adapt this algorithm to the 
land-cover segmentation task, where the learning process from only partially available inputs is necessary. This is due to missing measurements 
which are mainly caused by cloud or snow cover. 
Our approach involves combining a reasonably sized model inspired by U-net architecture and its skip connections\cite{unet-2015} with hierarchical VAEs, enabling
us to obtain a latent space distribution that represents the desired segmentation. Additionally, this setup will allow us to generate both segmentation from images 
and images from segmentation.

We will compare our approach with MixMatch \cite{mixmatch-2019} algorithm, which unifies the consistency regularization methods with proxy-labeling
methods \cite{ssl-overview-2020} and is the cornerstone for other algorithms such as ReMixMatch \cite{remixmatch-2020} and FixMatch \cite{fixmatch-2020}.
To our knowledge, the MixMatch algorithm has only been used for classification tasks (CIFAR10, SVHN). We will adapt it, so that is applicable
to the segmentation task. To verify its suitability and compare it with our method, we will use CityScape dataset and standard metrics used in 
segmentation, such as intersection over union (IoU), plain accuracy  and others. We will also validate the generative capabilities of the novel approach (e.g. 
inpainting of missing data).
\todo{Rewrite it once the actual evaluation is done.}

In chapter \ref{motivation-theory} we provide a brief introduction to SSL and its methods.  We will discuss the main ideas 
and cornerstones of MixMatch and the novel symmetrical equilibrium learning, namely consistency regularization with proxy 
labelling methods and VAE framework respectively. 
We also introduce the hieachical models (HVAE) and ladder variational autoencoders (LVAE) in this chapter as a necessary foundation for 
the novel algorithm. The description of the segmentation task, dataset, metrics and models architecture is provided
in chapter \ref{methods}. The experiments and results are available in \ref{results} with a summary of main findings and contributions in \ref{conclusions}. 
We make all code publicly available \footnote[2]{At \url{https://github.com/hruskan1/SSL-diploma-thesis}}. 


