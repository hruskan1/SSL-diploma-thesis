% Background (1-2 pages)

% Introduction to the topic and its importance
% Explanation of the real-world problem or scenario that the thesis aims to address
% Overview of the current state of research and any recent advances in the field
% Discussion of any challenges or limitations of existing approaches to the problem or scenario

Satellite imagery provides a wealth of information about our planet and has become a standard tool for monitoring, predicting and understanding 
the change in vegetation, agriculture and human environmental impact. With advancements in satellite technology, it is now possible to collect
massive amounts of high-resolution multispectral images over time, enabling researchers to perform complex spatio-temporal analyses of these 
datasets. However, the sheer magnitude of the raw data makes it challenging to process and analyze effectively. Moreover the remote sensing 
still has to face other challenges, which are rare in other areas of computer vision. Those are among others partial measurements 
(i.e. cloud cover, electro-magnetic (EM) interference), geolocation, different quality of measurement (spatial resolution, different EM bands) and 
calibration (issue of atmospheric reflectance).

From a machine learning perspective, the main challenges are the large amounts of unannotated data and partially missing measurements.
To tackle the former, semi-supervised learning \cite{ssl-book-2006} (SSL) have shown promise in leveraging the abundance of unlabeled
data to improve model performance. SSL algorithms aim to learn from labeled and unlabeled data, using the labeled data to guide the 
learning process and the unlabeled data to regularize the model. As for the issue of partially missing measurements, generative
models, such as variational autoencoders (VAEs) \cite{intro-vae-2019}, have shown potential to fill in the missing data gaps. 
By learning a generative model of the data distribution, VAEs can be used to impute missing values in a dataset, making it possible to 
fully utilize the available data.

The thesis topic is motivated by the real-world problem of segmenting national park's satellite imagery to monitor and predict 
its forest development. The forest development prediction could enable national park rangers to respond proactively to protect forest vegetation and thus
improve the national park protection. The thesis aims to create an approach that could be used in preprocessing to obtain segmentation for many images and 
generate inpainting for partially missing measurements. The project is funded by European Space Agency (ESA)
\footnote[1]{\url{https://eo4society.esa.int/projects/spatiotemporal-sen2vhr/}}.

The authors of \cite{not-published} proposed a novel SSL algorithm that employs symmetrical learning with hierarchical VAEs, which is capable of
handling exponential families of distribution, not just multivariate Gaussians for the latent variable. We adapt this algorithm for  
land-cover segmentation task, where the learning process from only partially available inputs is necessary. This is due to the missing measurements 
which are mainly caused by cloud or snow cover. 
Our approach involves combining a reasonably sized model inspired by U-net architecture and its skip connections\cite{unet-2015} with hierarchical VAEs, enabling
us to obtain a latent space distribution that represents the desired segmentation. Additionally, this setup will allow us to generate both segmentation from images 
and images from segmentation.

We will compare our approach with MixMatch \cite{mixmatch-2019} algorithm, which unifies the consistency regularization methods with proxy-labeling
methods \cite{ssl-overview-2020} and is the cornerstone for other algorithms such as ReMixMatch \cite{remixmatch-2020} and FixMatch \cite{fixmatch-2020}.
To our knowledge, the MixMatch algorithm has only been used on classification tasks (CIFAR10, SVHN). We will adapt it so it is applicable
to the segmentation task. To verify its suitability and compare it with our method, we will use CityScape dataset and standard metrics used in 
segmentation, such as intersection over union (IoU), plain accuracy  and others. We will also validate the generative capabilities of the novel approach (e.g. 
inpainting of missing data).
\todo{Rewrite it once the actual evaluation is done.}

We provide a brief introduction to SSL and its methods in chapter \ref{motivation-theory}. We will discuss the main ideas and cornerstones of 
MixMatch and our approach, namely consistency regularization, proxy label methods and VAE. In \ref{methodology}, we describe the dataset and 
segmentation task and provide the definition of the evaluation metrics. We also introduce the ladder variational autoencoders (LVAE) and 
symmetrical learning. Furthermore, we provide a description of MixMatch algorithm. The experiments and results are available in 
\ref{experiments-results} with a summary of main findings and contributions in \ref{conclusions}. We make all code publicly available
\footnote[2]{At \url{https://github.com/hruskan1/SSL-diploma-thesis}}. 

We would also like to draw the reader's attention to the fact that we have used Natural Language Processing (NLP) models (or tools that use them) 
in parts of this thesis. The details and extent of their use is described in the declaration on page \pageref{sec:decleration}.


% Objectives (1-2 paragraphs)

% Explanation of the specific objectives of the thesis
% Description of the research methods and approaches that will be used to achieve these objectives
% Brief overview of the expected results or outcomes of the research


% Thesis Structure (1-2 paragraphs)

% Brief overview of the structure of the thesis
% Explanation of how the different sections of the thesis contribute to answering the research question and achieving the objectives


% To make it longer, you could expand on the importance and potential impact of accurately monitoring and predicting changes in forest development,
% as well as the challenges and limitations of traditional methods. You could also provide more background on SSL, including its history and 
% previous applications in different fields. Additionally, you could discuss the potential advantages and disadvantages of combining 
% spatio-temporal U-nets with variational autoencoders and how they complement each other in your approach. Finally, you could elaborate on the 
% evaluation metrics and their significance in assessing the performance of your approach.