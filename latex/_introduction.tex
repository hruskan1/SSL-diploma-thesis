% Background (1-2 pages)

% Introduction to the topic and its importance
% Explanation of the real-world problem or scenario that the thesis aims to address
% Overview of the current state of research and any recent advances in the field
% Discussion of any challenges or limitations of existing approaches to the problem or scenario

Satellite imagery provides a wealth of information about our planet and has become a standard tool for monitoring, predicting
and understanding the change in vegetation, agriculture and human environmental impact. With advances in satellite technology, it
is now possible to collect considerably large amounts of high-resolution multispectral images over time, enabling researchers to perform
complex spatiotemporal analyses of these datasets. However, the magnitude of the raw data makes it challenging to process and analyze it effectively. 
Moreover, remote sensing still faces further challenges, which are rare in other areas of computer vision. Those are, in particular, partial
measurements (i.e. cloud cover, electromagnetic (EM) interference), geolocation, different quality of measurement (spatial resolution, different
EM bands) and calibration (issue of atmospheric reflectance).

From a machine learning perspective, the main challenges are considerable amounts of unannotated data and partially missing measurements.
To address the first, semi-supervised learning (SSL) \cite{ssl-book-2006}  has shown promise in leveraging the abundance of unlabeled
data to improve model performance. SSL algorithms are designed to learn from labeled and unlabeled data, using the labeled data to guide the 
learning process and the unlabeled data to regularize the model. Regarding the issue of partially missing measurements, generative
models, such as variational autoencoders (VAEs) \cite{intro-vae-2019}, have shown the potential to fill in the missing data gaps. 
By learning a generative model of the data distribution, VAEs can be used to impute missing values in a dataset, making it possible to 
fully utilize the available data.

The thesis topic is motivated by the real-world problem of segmenting the satellite imagery of a national park to monitor and predict 
its forest development. The forest development prediction could allow national park rangers to respond proactively to protect forest 
vegetation and thus improve national park preservation. The thesis aims to create an approach that could be used in data preprocessing 
to obtain segmentation for many images and generate inpainting for partially missing measurements. The project is funded by the European
Space Agency (ESA) \footnote[1]{\url{https://eo4society.esa.int/projects/spatiotemporal-sen2vhr/}}.

The authors of \cite{sym-learning-2023} proposed a novel SSL algorithm that employs symmetrical learning with hierarchical VAEs, capable of
handling exponential families of distribution, not just multivariate Gaussians for the latent variable. We adapt this algorithm to the 
land-cover segmentation task, where the learning process from only partially available inputs is necessary. This unavailability issue is
due to missing measurements, which are mainly caused by clouds or snow cover. Our approach combines a reasonably sized model inspired by the
U-net architecture and its skip connections\cite{unet-2015} with hierarchical VAEs, Such an approach would enable us to obtain a latent space
distribution that represents the desired segmentation. Additionally, this setup will allow us to generate both segmentation from images and 
images from segmentation.

We compare our approach with the MixMatch \cite{mixmatch-2019} algorithm, which unifies consistency regularization methods with proxy-labeling
methods \cite{ssl-overview-2020} and is the cornerstone for other algorithms such as ReMixMatch \cite{remixmatch-2020} and FixMatch \cite{fixmatch-2020}.
To our knowledge, the MixMatch algorithm has only been used for classification tasks (CIFAR10, SVHN). We adapt it so that it applies
to the segmentation task. To verify its suitability and compare it with our method, we use the CityScape dataset and standard metrics used in 
segmentation, such as plain accuracy, intersection over union (IoU), and others. We also validate the generative capabilities of the novel approach 
(e.g.,~inpainting of missing data).

We briefly introduce SSL and its methods in chapter \ref{motivation-theory}. We discuss the main ideas 
and cornerstones of MixMatch and the novel symmetrical equilibrium learning, namely consistency regularization with proxy 
labelling methods and VAE framework, respectively. This chapter also provides the hierarchical models (HVAE) and ladder
variational autoencoders (LVAE) as a foundation for the novel algorithm. The segmentation task, dataset, metrics and models architecture
are described in chapter \ref{methods}. The experiments and results are available in chapter \ref{results} with a summary of the main findings and 
contributions in \ref{conclusions}. We make all code publicly available \footnote[2]{At \url{https://github.com/hruskan1/SSL-diploma-thesis}}.
