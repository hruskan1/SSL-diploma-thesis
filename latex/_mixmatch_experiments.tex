\section{Mixmatch experiments}
As the original implementation of MixMatch is written in \texttt{TensorFlow} we have re-implemented it in \texttt{PyTorch}. To verify the correctness of our implementation, we have run tests on CIFAR10 and compared our results with those reported in \cite{mixmatch-2019} and \cite{wide-resnet-2017}. The 
comparison is shown in table \ref{tab:mixmatch-cifar10}. For details of reported results, see respecitve papers. 
The specific hyperparameters of experiments are provided with the code and are available. We will not report them here.
 
\begin{table}[htb]
    \begin{tabular}{|c|c|c|c|c|c|c|}
    \hline
    Labels [\#]  & 250 & 500 & 1000 & 2000 & 4000 & All \\
    \hline
    Our code & $88.45$ & $89.58$& $91.83$ & $93.03$ & $93.50$ & $93.54$\\
    \hline
    Reported & $88.92\pm0.87$ & $90.35\pm0.94$ & $92.25\pm0.32$ & $92.97\pm0.15$ & $93.76\pm 0.06$ & $94.27$\\
    \hline 
    Baseline & $38.42$ & $45.77$ & $50.42$ & $60.21$ & $79.57$ &  \\
    \hline
    \end{tabular}
    \caption[Mixmatch accuracy on CIFAR10]{Mixmatch accuracy rate (\%) on CIFAR10 dataset \cite{cifar10-2009}. Labels row corresponds to number of labeled points available during trainig. 
    The last column ("All") corresponds to fully-supervised mode performance.
    We additionally provide the supervised row, which contains the supervised baseline trained on given number of images (Our code).
    Our results are based only on one run.}
    \label{tab:mixmatch-cifar10}
\end{table}

We have 
\todo{Write about the selection of the categories, etc.}
     