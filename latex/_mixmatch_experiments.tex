%% 
As the original implementation of MixMatch is written in \texttt{TensorFlow} and supports only the classification task only, we have re-implemented it in \texttt{PyTorch}
and adapted it to the task of image segmentation. To verify the correctness of our implementation, we have run tests on CIFAR10 and compared our results with those reported in \cite{mixmatch-2019} and \cite{wide-resnet-2017}. The 
comparison is shown in table \ref{tab:mixmatch-cifar10}. For details of reported results, see respecitve papers. Our results are based only on one run. 
\todo{Report the sensitivity with hyperparameters. And the training of supervised model}
\begin{table}[htb]
    \begin{tabular}{|c|c|c|c|c|c|c|}
    \hline
    Labels  & 250 & 500 & 1000 & 2000 & 4000 & All \\
    \hline
    Our code & $88.45$ & $89.58$& $91.83$ & $93.03$ & $93.50$ & $93.54$\\
    \hline
    Reported & $88.92\pm0.87$ & $90.35\pm0.94$ & $92.25\pm0.32$ & $92.97\pm0.15$ & $93.76\pm 0.06$ & $94.27$\\
    \hline
    \end{tabular}
    \caption[Mixmatch accuracy on CIFAR10]{Mixmatch accuracy rate (\%) on CIFAR10 dataset \cite{cifar10-2009}. Labels row corresponds to number of labeled points available during trainig. 
    The last column ("All") corresponds to fully-supervised mode performance.}
    \label{tab:mixmatch-cifar10}
\end{table}
     