\section{Problem definition}
Our main objective is to develop an effective approach for segmenting satellite imagery using a limited number of 
accessible images and a smaller number of expert-provided annotations. The segmentation task involves assigning a 
specific label to each pixel in the image, effectively partitioning the image into distinct regions based on their 
semantic or visual characteristics.

To achieve this objective, we will employ the U-Net architecture, which has been proven to be effective for image 
segmentation~\cite{unet-2015}. The U-Net architecture will serve as the core model for evaluating the proposed algorithms.
We will use the plain accuracy and average ``intersection over union'' (IoU) metrics to evaluate the 
quality of the segmentation. The IoU measures the overlap between the predicted segmentation and the ground truth 
segmentation, i.e.:
$$
\mathrm{IoU} = \frac{\text{area of intersection}}{\text{area of union}},
$$
where the area of intersection corresponds to the area where the model predicts the given class and the class is in the ground truth segmentation.
The union area is then composed of an area where either model predicts the class or the class is in the ground truth segmentation (or both).
This can be rephrased to binary classification terminology as: 
$$
\mathrm{IoU} = \frac{\mathrm{TP}}{\mathrm{TP}+\mathrm{FP}+\mathrm{FN}},
$$
where the $\mathrm{TP}$ is \textit{true positive}, $\mathrm{FP}$ is \textit{false positive} and $\mathrm{FN}$ is \textit{false negative}. 
The average IoU is then the simple average of the class IoU over all classes. 

To evaluate our methods and ensure reproducible experiments, we will utilize the CityScape dataset~\cite{cityscapes-2016} as a benchmark. The CityScape dataset is almost publicly available, with a fee-less 
registration required to access. Although the dataset primarily consists of urban street scenes, we believe that the complexity of segmentation in this dataset is comparable to, or even higher than, the 
challenges encountered in land-coverage segmentation of forests.

By leveraging the CityScape dataset and employing the U-Net architecture, we can thoroughly evaluate our methods and 
provide meaningful results that others can replicate and validate.

The U-Net is a feed-forward convolutional neural network (FF CNN), and its architecture consists of an encoder and a decoder part, which work together for image segmentation tasks. 
(We would like to point out to the reader that in this context, the  ``encoder'' and ``decoder'' are not referring to the VAE networks but to the structure of the U-Net architecture):
\begin{itemize}
    \item \textit{Encoder}: The encoder part of U-Net is responsible for downscaling the spatial resolution of the input
     image while increasing its channel capacity. This is achieved through a series of convolutional blocks and downsampling. 
     Each block typically consists of one or more convolutional layers, followed by non-linear activation
    functions (such as ReLU) and batch normalization. The encoder acts as a feature extractor for image
    classification tasks.
    \item \textit{Decoder}: The decoder part of U-Net is connected to the encoder and performs the opposite operation
     of the encoder. It upscales the feature maps while reducing the number of channels. This is done through 
     a series of upsampling and convolutional blocks. Like the encoder, each block consists of convolutional
    layers, activation functions, and batch normalization. 

    \item \textit{Skip Connections}: U-Net incorporates skip connections between the encoder and decoder. These 
    connections allow information from the encoder to be directly propagated to the corresponding decoder 
    block at the same spatial resolution. By sharing this information, U-Net helps to preserve spatial details
     and enables more precise segmentation.

    \item \textit{Convolutional Blocks}: The convolutional blocks in U-Net typically consist of convolutional layers, activation 
    functions, and batch normalization. The number of convolutional layers in each block can vary but is usually up to 2 or 3. 
    These blocks are responsible for learning and extracting features from the input data. Their parameters are typically chosen such
    that they retain the spatial resolution

    \item \textit{Spatial Resolution}: U-Net can maintain the spatial resolution of the input or perform downsampling and upsampling 
    operations at specific layers, depending on the chosen parameters. This flexibility allows U-Net to
     capture both local and global information.
\end{itemize}
Overall, the U-Net architecture is designed to effectively capture contextual information and spatial details for image segmentation tasks. It has been widely used and has shown promising results in various applications.